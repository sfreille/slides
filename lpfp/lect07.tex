\documentclass[handout,final,xcolor=dvipsnames]{beamer}

\useinnertheme{rectangles}
\usecolortheme[RGB={140,206,150}]{structure}
\useoutertheme{infolines}
%\setfootline{\insertinstitute,\insertauthor, \hfill slide \insertframenumber/\insertotalframenumber}
\usetheme[height=12mm]{Darmstadt}

\definecolor{lightblue}{RGB}{217,244,212}

\setbeamertemplate{blocks}[rounded]

%\setbeamercolor{block title}{fg=White,bg=Red}
\setbeamercolor{block body}{bg=lightblue}

%\useoutertheme[footline=authortitle,subsection=false]{miniframes}
%\usepackage{pgf,pgfpages}
%\pgfpagesuselayout{resize to}[a4paper,landscape,border shrink=5mm]
\usepackage{xcolor}
\usepackage{booktabs}
\usepackage{fontenc}
\usepackage[utf8]{inputenc}
\usepackage{hyperref}
\hypersetup{colorlinks=false,
linkcolor=blue,
citecolor=red,
urlcolor=blue}


\author{Sebastián Freille (UNC-UCC)}
\title{Universidad Nacional de La Plata (UNLP) \\
  Maestría en Finanzas Públicas Provinciales y Municipales \\ La
  Política de las Finanzas Públicas \\ Clase 8-9}
\date{}
\institute{}


\AtBeginSection[]{
    \begin{frame}
    \vfill
    \centering
    \begin{beamercolorbox}[sep=8pt,center,shadow=true,rounded=true]{title}
        \usebeamerfont{title}\insertsectionhead\par%
    \end{beamercolorbox}
    \vfill
    \end{frame}
}


\begin{document}
\maketitle

\section{Clientelismo}

\begin{frame}\frametitle{Clientelismo}
  \begin{itemize}\itemsep 10pt
      \item Clientelismo suele estar asociado a la existencia de
        ciertos modelos económicos $\longrightarrow$ tradicionalmente
        en Latinoamérica, se postulaba que el clientelismo coexistía
        con modelos populistas
        \item Sin embargo, en contra de algunas expectativas teóricas,
          el clientelismo no disminuyó significativamente con la
          adopción de modelos neoliberales y de mercado
    \end{itemize}
  \end{frame}


  \begin{frame}\frametitle{Clientelismo como fenómeno}
  \begin{itemize}\itemsep 10pt
      \item Puede pensarse al clientelismo como una forma de
        redistribución $\longrightarrow$ en muchos países, la
        redistribución implica alguna forma de empleo público
        \item En el sur de Italia, por ejemplo, sólo la mitad de los
          salarios en el empleo público se explica como redistribución
          de este tipo [Alesina et al (2001)]
          \item En América Latina, el tamaño del sector público medido
            en términos de empleo es mayor donde hay mas clientelismo
            [Calvo and Murillo (2004)]
                \end{itemize}
  \end{frame}



  \begin{frame}\frametitle{Clientelismo como contrato}
  \begin{itemize}\itemsep 10pt
      \item La política redistributiva puede ser vista como una
        relación de intercambio $\longrightarrow$ los políticos (candidatos) tienen un problema de compromiso porque ellos no
        quieren implementar las políticas \textit{ex post} que induzca
        a los votantes a votar por el/ella
        \item El político, ya electo o candidato, desea ofrecer
          políticas a los ciudadanos \textit{ex ante} para que ellos
          le retribuyan con su voto
        \item Estos ``contratos'' no son legalmente exigibles por lo
          que para existir deben ser de cumplimiento autómatico
          \item Pero ambas partes tienen incentivos  a no cumplir y
            ofrecer políticas no óptimas
     
    \end{itemize}
  \end{frame}



  \begin{frame}\frametitle{Clientelismo como contrato (cont.)}
  \begin{itemize}\itemsep 10pt
      \item ¿Qué tipo de incentivos y favores se deberán ofrecer entre
        las partes para que el problema del compromiso desaparezca?
        \item Tanto el político como el votante debe poder
          ``castigar'' al otro.
          \item La propuesta de un trabajo en la burocracia es un tipo
            de incentivo que funciona como un metodo de redistribución
            creible, selectivo y reversible y que además ata la
            utilidad del votante al exito político de un determinado
            candidato $\longrightarrow$ redistribución compatible con
            incentivos
            
            \item Ejemplo: antes de las elecciones el partido
              demócrata cristiano repartiría un zapato izquierdo a
              cada uno de sus ``clientes'' prometiendo el zapato
              derecho si resultaba reelecto. Tener un solo zapato es
              inútil para cada parte!
    \end{itemize}
  \end{frame}



   \begin{frame}\frametitle{Clientelismo como contrato (cont.)}
  \begin{itemize}\itemsep 10pt
      \item Empleo vs transferencias de ingresos --ie por qué el
        primero es creible? $\longrightarrow$
        primero, un contrato de empleo público provee una renta al
        trabajador; segundo, porque es preferible antes que sufrir el
        costo de subir impuestos; tercero, si el comportamiento
        político es observable, puede eliminarse el empleo (pensemos
        empleo por contrato)
        \item Esta es la razón por la que otras políticas públicas
          estan sub-provistas en equilibrio (transferencias de
          ingreso por no creibles o bienes públicos/obra pública
          porque cambian los términos de intercambio entre políticos y
          votantes
          \item Finalmente, muestran que características como baja
            productividad y desigualdad hacen del clientelismo
            (mecenezago) una herramienta atractiva. 
    \end{itemize}
  \end{frame}


  
  
  \begin{frame}\frametitle{Clientelismo: Mecenazgo)}
  \begin{itemize}\itemsep 10pt
      \item Calvo and Murillo (2004) estudian el caso argentino y
        modelan el ``mecenazgo'' $\longrightarrow$ práctica de ofrecer
        trabajos en el Estado a cambio de compra de voluntad electoral
        \item Ante la ausencia de reglas para el acceso al servicio
          civil, el empleo público puede verse como un \textbf{bien
            público excluible}  que puede distribuirse a distritos
          partidarios.
          \item En este sentido, el mecenazgo puede actuar como un
            mecanismo para fortalecer la coalición
            político-electoral. 
    \end{itemize}
  \end{frame}


 
  \begin{frame}\frametitle{Clientelismo: Mecenazgo (cont.)}
  \begin{itemize}\itemsep 10pt
      \item Calvo and Murillo (2004) estudian el caso argentino y
        modelan el ``mecenazgo'' $\longrightarrow$ práctica de ofrecer
        trabajos en el Estado a cambio de compra de voluntad electoral
        \item Ante la ausencia de reglas para el acceso al servicio
          civil, el empleo público puede verse como un \textbf{bien
            público excluible}  que puede distribuirse a distritos
          partidarios.
          \item En este sentido, el mecenazgo puede actuar como un
            mecanismo para fortalecer la coalición
            político-electoral. 
    \end{itemize}
  \end{frame}


  
  \begin{frame}\frametitle{Clientelismo: Mecenazgo (cont.)}
  \begin{itemize}\itemsep 10pt
      \item Una pregunta interesante, sin embargo, es por qué los
        partidos tienen diferentes preferencias en relación al gasto
        en mecenazgo.
        \item Si el acceso a los recursos fiscales está desigulamente
          distribuido entre los partidos y/o las habilidades
          desigualmente distribuidas entre los votantes, el mecenazgo
          no tendrá el mismo retorno electoral para los diferentes
          patrones (partidos).
              \end{itemize}
  \end{frame}
  


 \begin{frame}\frametitle{Clientelismo: Mecenazgo (cont.)}
\begin{figure}[!htbp]
  \centering \vspace{-0.5cm}
 \caption{Diferencias de habilidades y prima salarial}
  \label{fig:lic1}
  \includegraphics[scale=0.3]{votebuying3}
 \end{figure}
\end{frame}



 \begin{frame}\frametitle{Clientelismo: Mecenazgo (cont.)}
\begin{figure}[!htbp]
  \centering \vspace{-0.5cm}
 \caption{Voto peronista y acceso a recursos fiscales (gasto público)}
  \label{fig:lic1}
  \includegraphics[scale=0.3]{votebuying4}
 \end{figure}
\end{frame}


  

    \begin{frame}\frametitle{Clientelismo como compra de votos}
  \begin{itemize}\itemsep 10pt
      \item Un trabajo sobre clientelismo en Argentina es el de
        Brusco, Nazareno and Stokes (2004). Estudian el clientelismo
        como ``compra de votos'' $\longrightarrow$ la entrega de
        prestaciones directas (en efectivo o especie) como
        contraprestación al voto
        \item Utilizan datos de encuestas especialmente diseñadas para
          capturar ese fenómeno. 
    \end{itemize}
  \end{frame}


    \begin{frame}\frametitle{Clientelismo como compra de votos (cont.)}
  \begin{itemize}\itemsep 10pt
  \item Se hacen cuatro preguntas:
    \begin{itemize}\itemsep 5pt \medskip
    \item ¿Cuán extendido es la compra de votos en Argentina?
    \item ¿Cuán efectiva es la compra de votos?
      \item ¿Qué tipos de votantes son más proclives a ``vender'' sus
        votos?
        \item ¿Por qué la compra de votos funciona, a pesar del
          secretismo del voto. 
        \end{itemize}
        \item En otras palabras, ¿que hace que un votante no reciba
          con una mano y vote con la otra?
    \end{itemize}
  \end{frame}


  
    \begin{frame}\frametitle{Clientelismo como compra de votos (cont.)}
  \begin{itemize}\itemsep 10pt
  \item Realizaron un estudio mixto de encuestas poblacionales en 2001 y 2002 junto
    con entrevistas en profundidad (en persona).
    \item Entre 12\% (piso) y 35\% (techo) de los respondentes (Buenos Aires,
      Córdoba, Misiones) declaran haber recibido algún tipo de bien
      y/o pago en dinero.
      \item Correlación negativa entre ingresos y disposición a
        aceptar dadiva clientelar
        \item Los que más se inclinan a aceptar dadivas son aquellos
          más jovenes y con simpatías peronistas (que ingresaron al
          electorado durante la epoca de Menem).
          \item Visión minoritaria $\longrightarrow$ clientelismo y
            compra de votos en países pobres como una suerte de ``Estado
            de bienestar'' de los más pobres
    \end{itemize}
  \end{frame}


   \begin{frame}\frametitle{Clientelismo como compra de votos (cont.)}
\begin{figure}[!htbp]
  \centering \vspace{-0.5cm}
 \caption{Respondentes a pregunas sobre ayuda y relaciones clientelares}
  \label{fig:lic1}
  \includegraphics[scale=0.3]{votebuying1}
 \end{figure}
\end{frame}



 \begin{frame}\frametitle{Clientelismo como compra de votos (cont.)}
\begin{figure}[!htbp]
  \centering \vspace{-0.5cm}
 \caption{Efecto de recibir dádiva sobre probabilidad de voto}
  \label{fig:lic1}
  \includegraphics[scale=0.3]{votebuying2}
 \end{figure}
\end{frame}



 \begin{frame}\frametitle{Clientelismo como compra de votos (cont.)}
   \begin{itemize}\itemsep 10pt
     \item A pesar de que pueden observar muchas cosas, los punteros
       no pueden observar directamente la decisión de cada votante
       --pero tienen proxies. 
  \item Los autores encuentran tres canales por los que el
    clientelismo sobrevive el secretismo del voto: 1) razón
    quid-pro-quo (votan porque temen perder el beneficio); 2) razón
    normativa (cumplen su parte porque obligación moral; 3) razón
    incertidumbre (valoran mucho mas cualquier monto pequeño presente
    que promesa mayor futura)
    \item En Argentina, el clientelismo no parece ser un fenómeno tan
      extendido --al menos por la propia declaración de quienes
      responde (problema de ``self-incrimination''). 
  \end{itemize}
  \end{frame}

  


   \begin{frame}\frametitle{Venezuela y la base Maisanta}
  \begin{itemize}\itemsep 10pt
      \item En 2004, el régimen de Chávez distribuyó internamente la
        base de firmantes de las peticiones para los referendos
        revocatorios en 2002 y 2003.
        \item Los firmantes de la última petición que condujo al
          referendum revocatorio (que Chavez ganó) fue integrada en un
          base de datos estadística conocida como \textit{base
            Maisanta}.
          \item Jatar (2006) documenta una gran cantidad de historias
            de personas que perdieron sus trabajos por figurar como
            firmantes. Hsieh et al (2011) encuentran evidencia
            sistemática de castigo del oficialismo a votantes
            opositores $\longrightarrow$ aquellos identificados como
            opositores sufrieron una caída en ingresos de 5\% y una
            caída de 1.3\% en las tasas de empleo comparando el antes
            y el después de la liberación de la base. 
     \end{itemize}
  \end{frame}



  \begin{frame}\frametitle{Políticas redistributivas: Implementacion}
  \begin{itemize}\itemsep 10pt
  \item A diferencia de políticas de prevención de la corrupción
    --generalistas, universales-, otras políticas tienen beneficiarios
    y perjudicados. En este contexto, las políticas redistributivas
    son un caso ejemplar
    \item Pueden existir incentivos asimetricos en la implementacion
      de estos programas para que implementadores y diseñadores tengan
      diferentes motivos.
      \item Uno de los aspectos centrales son los beneficios
        electorales en la implementacion de programas de
        transferencias monetarias condicionadas (CCTs). 
    \end{itemize}
\end{frame}




\begin{frame}\frametitle{Políticas redistributivas: Implementacion (cont.)}
  \begin{block}{Progresa-Oportunidades en Mexico (Bauer, 2017)}
En un trabajo importante, Bauer estudia los incentivos políticos de
los implementadores de un programa de redistribución. Focaliza en dos
incentivos: 1) electorales sobre todo cerradas; 2) alineación con el
gobierno nacional. Muestra que la
registracion de nuevos beneficiarios del programa fue
significativamente mayor en las elecciones previas a la reforma del
\textit{blindaje electoral} en 2001 que incrementó las barreras a la
interferencia política. También encuentra que este patrón fue mas
acentuado en aquellos municipios mas disputados electoralmente. No
sólo eso, sino que también encuentra que la registracion adicional
debida a incentivos políticos del implementador empeoro el
``targeting'' y la eficiencia general del programa
\end{block}
\end{frame}


\begin{frame}\frametitle{Políticas redistributivas: Implementación
    (cont.)}
  \begin{figure}[htbp]
    \centering \vspace{0cm}
    \includegraphics[scale=0.4]{figura3_1}
    \caption[Modelo de etapas]{Expansión del programa Progresa-Oportunidades}
    \label{fig:figura1}
  \end{figure}
\end{frame}



\begin{frame}\frametitle{Políticas redistributivas: Implementación
    (cont.)}
  \begin{figure}[htbp]
    \centering \vspace{0cm}
    \includegraphics[scale=0.5]{figura3_2}
    \caption[Modelo de etapas]{Resultados}
    \label{fig:figura1}
  \end{figure}
\end{frame}



\begin{frame}\frametitle{Políticas redistributivas: Implementacion (cont.)}
  \begin{itemize}\itemsep 10pt
  \item El autor compara como la evolución de los principales
    indicadores sociales --pobreza alimentaria, capacidades,
    patrimonio y Gini- evolucionaron entre municipalidades con
    implementación en años electorales vs no electorales
    \item Los resultados muestran que la registración adicional de
      beneficiarios en años electorales no causan ningún impacto de
      bienester en terminos de pobreza o desigualdad
      \item El ``targeting'' que se realiza en años electorales es mas
        ineficiente que el que se realiza en años no electorales y
        conduce a menor efectividad del programa por beneficiario
    \end{itemize}
\end{frame}


\begin{frame}\frametitle{Políticas redistributivas: Implementación
    (cont.)}
  \begin{figure}[htbp]
    \centering \vspace{0cm}
    \includegraphics[scale=0.5]{progresa01}
    \caption[Modelo de etapas]{Resultados}
    \label{fig:figura1}
  \end{figure}
\end{frame}




\end{document}



