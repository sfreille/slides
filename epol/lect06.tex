\documentclass[handout,final,xcolor=dvipsnames]{beamer}

\useinnertheme{rectangles}
\usecolortheme[RGB={140,206,150}]{structure}
\useoutertheme{infolines}
%\setfootline{\insertinstitute,\insertauthor, \hfill slide \insertframenumber/\insertotalframenumber}
\usetheme[height=12mm]{Darmstadt}

\definecolor{lightblue}{RGB}{217,244,212}

\setbeamertemplate{blocks}[rounded]

%\setbeamercolor{block title}{fg=White,bg=Red}
\setbeamercolor{block body}{bg=lightblue}

%\useoutertheme[footline=authortitle,subsection=false]{miniframes}
%\usepackage{pgf,pgfpages}
%\pgfpagesuselayout{resize to}[a4paper,landscape,border shrink=5mm]
\usepackage{xcolor}
\usepackage{booktabs}
\usepackage{fontenc}
\usepackage[utf8]{inputenc}
\usepackage{hyperref}
\hypersetup{colorlinks=false,
linkcolor=blue,
citecolor=red,
urlcolor=blue}


\author{Sebastián Freille (UNC-UCC)}
\title{Universidad Nacional de La Plata (UNLP) \\
  Maestría en Finanzas Públicas Provinciales y Municipales \\ La
  Política de las Finanzas Públicas \\ Clase 7-8}
\date{}
\institute{}


\AtBeginSection[]{
    \begin{frame}
    \vfill
    \centering
    \begin{beamercolorbox}[sep=8pt,center,shadow=true,rounded=true]{title}
        \usebeamerfont{title}\insertsectionhead\par%
    \end{beamercolorbox}
    \vfill
    \end{frame}
}


\begin{document}
\maketitle


\section{Grupos de interés y canales de influencia}

\begin{frame}\frametitle{El rol de los grupos de interés}
\begin{itemize}\itemsep 15pt
\item En una sociedad (democrática) ideal, la política económica es determinada a
  través de la fórmula ``un hombre, un voto''.
\item En las sociedades (democráticas) reales, los grupos de intereses
  especiales juegan un rol importante en el proceso de determinación
  de la política económica.
\item Los grupos de interes (GIS) o grupos de presión representan
  intereses acotados, estrechos (cámaras, sector arrocero). Pero también pueden representar
  intereses generales, amplios (jubilados, capitalistas, ambientalistas)
\item Existen varias formas de influencia de los GIS en la política
  económica.
\end{itemize}
\end{frame}


\begin{frame}\frametitle{Canales de influencia}
\begin{itemize}\itemsep 15pt \medskip
\item Existen varios posibles canales de --formas de ``comprar''-- influencia sobre la politica
  económica:
\begin{itemize}\itemsep 15pt \medskip 
\item A través de contribuciones/donaciones a campañas políticas
\item A través de pagos de coimas/sobornos a funcionarios y/o
  políticos en el Gobierno
\item A través de acciones directas (paros, movilizaciones, despidos)
\item A través de manifestar apoyo público (``endorsement'') de
  ciertos candidatos
\end{itemize}
\item Los instrumentos de influencia no necesariamente se miden en
  dinero o en valores.
\end{itemize}
\end{frame}


\begin{frame}\frametitle{Midiendo la actividad de los GIS}
\begin{itemize}\itemsep 15pt
\item Resulta bastante difícil capturar y medir la actividad de los
  GIS por varias razones:
\begin{itemize}\itemsep 15pt \medskip
\item Deficientes marcos legales y regulatorios
\item Actividad ligada con la transparencia de intereses económicos
\item Falta de control y sanciones
\end{itemize}
\item Algunas iniciativas ayudan: https://www.opensecrets.org/
\end{itemize}
\end{frame}


\section{Financiamiento político/electoral}

\begin{frame}\frametitle{¿Por qué estudiar financiamiento político?}
\begin{itemize}\itemsep 15pt
\item Analizar y explicitar las relaciones entre la forma de
  financiamiento y sus efectos --económicos y políticos-- tema central
  de la economía política
\item Diseño institucional (legal) $\longrightarrow$ puede no ser neutral y
  favorecer a partidos grandes y disminuir niveles de transparencia.
\item Permite aproximarse de manera más sistemática a un aspecto
  central de la corrupción política
\end{itemize}
\end{frame}

\begin{frame}\frametitle{Caso: lobby financiero y crisis 2007}
\begin{itemize}\itemsep 15pt
\item Entre las causas de las crisis financiera de 2007 se cita en
  ocasiones el rol de los grupos de interés asociados con el sector
  bancario y financiero. 
\item En el período previo a la crisis, hubo muchos donantes a
  campañas para favorecer legislación laxa e impedir legislacion
  restrictiva. 
\end{itemize}
\begin{table}[htbp]
  \centering
  \begin{tabular}[htbp]{lccc}
    Nombre	&	Periodo	&	Monto (mill)		Motivo		\\ \hline
Ameriquest Mortgage	&	2002-06	&	20.5		pasa
                                          ley		\\ \hline
Countrywide Financial	&	2002-06	&	8.7		pasar ley		\\ \hline
Citigroup	&	2002	&	3		no pasar ley		\\ \hline

  \end{tabular}
  \caption{Lobby del sector financiero en USA}
  \label{tab:4}
\end{table}
\end{frame}



\begin{frame}\frametitle{Caso: lobby financiero y crisis 2007 (cont.)}
\begin{figure}[htbp]\vspace{-0.5cm}
  \centering
  \includegraphics[scale=0.4]{lobbyingUS}
  \caption{Evolución de actividad de lobby}  \label{fig:5}
\end{figure}
\end{frame}

\begin{frame}\frametitle{Caso: lobby financiero y crisis 2007 (cont.)}
\begin{itemize}\itemsep 15pt
\item Igan et al (2012) en un excelente trabajo encuentran que los
  prestamistas (de creditos hipotecarios) más intensivos en lobby
\begin{itemize}\itemsep 5pt \meskip
\item dieron prestamos mas riesgosos (mayor ratio prestamo/ingreso)
\item securitizaron una proporción creciente de estos prestamos
\item obtuvieron rendimientos negativos (anormales) pre-crisis; pero
  positivos (anormales)
  post-rescate
\item tuvieron una probabilidad mayor de ser rescatados que
  prestamistas menos intensos en lobby. 
\end{itemize}
\item Conclusión $\longrightarrow$ los grupos de interés pueden
  \textbf{influenciar de manera crítica} las políticas adoptadas y
  tener impactos negativos sobre el bienestar individual y agregado. 
\end{itemize}
\end{frame}






\begin{frame}\frametitle{Financiamiento de campaña en Argentina}
\begin{itemize}\itemsep 15pt
\item Analizar y explicitar las relaciones entre la forma de
  financiamiento y sus efectos --económicos y políticos-- tema central
  de la economía política
\item Diseño institucional (legal) $\longrightarrow$ puede no ser neutral y
  favorecer a partidos grandes y disminuir niveles de transparencia.
\item Permite aproximarse de manera más sistemática a un aspecto
  central de la corrupción política
\item Discusión de futuro marco legal para lobbies y conflicto de
  intereses. 
\end{itemize}
\end{frame}


\begin{frame}\frametitle{Financiamiento de campaña en Argentina (cont.)}
\begin{itemize}\itemsep 15pt\item Financiamiento privado $\longrightarrow$ propuestas de políticas
  de partidos (candidatos) alineados con intereses específicos
  --i.e. alejados del mediano
\item Financiamiento público $\longrightarrow$ propuestas de políticas
  de partidos (candidatos) no alineados con ningún interés particular
  --i.e. coincidentes con el mediano
\item Las existencia de ambos tipos de financiamiento provoca fuerzas
  opuestas en el posicionamiento de las políticas.
\item Si definimos $CEPP=\frac{pri_{i,t}}{pub_{i,t}}$, entonces
  mientras mayor sea este ratio, mayor será la polarización en las
  políticas.
\end{itemize}
\end{frame}

\begin{frame}\frametitle{Financiamiento de campaña en Argentina (cont.)}
\begin{itemize}\itemsep 10pt
\item El tipo de financiamiento tambien puede estar relacionado con los votos
  obtenidos por los partidos. Dos casos:
\begin{itemize}\itemsep 15pt
\item Partidos simétricos ($p_1=p_2$) $\longrightarrow$ votos son independientes
  de contribuciones e iguales
\item Partidos no simétricos ($p_1 \neq p_2$) $\longrightarrow$ votos
  no son independientes de contribuciones --ejemplo cuando uno es el
  \textit{incumbent} y el otro es el \textit{challenger}.
\end{itemize}
\item Si el gasto en campaña influye sobre la popularidad de los
  partidos esto implicará diferentes porcentajes de votos para
  \textit{incumbents} y \textit{challengers}.
\item Esta diferencia (ventaja del oficialista)
  pareciera estar relacionada
  inversamente con el ratio  $CEPP=\frac{pri_{i,t}}{pub_{i,t}}$.
\end{itemize}
\end{frame}


\begin{frame}\frametitle{Financiamiento de campaña en Argentina (cont.)}
\begin{itemize}\itemsep 15pt
\item Partidos pueden recibir fondos públicos --suma fija para
  impresión de boletas más suma variable en función de votos previos-
  y fondos privados.
\item Hasta 2009, los partidos podían recibir fondos de
  personas y empresas. La ley 26571 prohibió aportes de empresas.
\item El régimen fue modificado tres veces
  durante el período 2003-2013. Los cambios mas relevantes fueron la
  prohibición de aportes de empresas, y la introducción de de tope de
  aportes de individuos y tope de gastos de campaña.
\end{itemize}
\end{frame}

\begin{frame}\frametitle{Financiamiento de campaña en Argentina (cont.)}
\begin{table}\centering\caption{Structure of campaign contributions:
    Argentina,2005-2013} \label{tab:struc1}
\begin{tabular}[htbp]{lcccccc}
  Source & 2005 & 2007 & 2009 & 2011 & 2013 & Period avg \\ \hline\hline
Public & 0.48 & 0.77 & 0.69 & 0.87 & 0.77 & 0.72 \\
Private & 0.23 & 0.17 & 0.27 & 0.12 & 0.20 &  0.20 \\
Other & 0.29 & 0.06 & 0.04 & 0.01 & 0.03 & 0.08 \\ \hline
Total & 1.00 & 1.00 & 1.00 & 1.00 & 1.00 & 1.00 \\
\end{tabular}
\end{table}
\end{frame}

\begin{frame}\frametitle{Financiamiento de campaña en Argentina (cont.)}
\begin{figure}[htbp]
  \centering
  \includegraphics[scale=0.25]{grafico1}
  \caption{Private funds as fraction of total funds}
  \label{fig:fig1}
\end{figure}
\end{frame}


\begin{frame}\frametitle{Financiamiento de campaña en Argentina (cont.)}
\begin{figure}[htbp]
  \centering
  \includegraphics[scale=0.32]{scatterplots2}
  \caption{Vote shares and private financing: Incumbents vs Challengers}
  \label{fig:scatterplots2}
\end{figure}
\end{frame}

\begin{frame}\frametitle{Financiamiento de campaña en Argentina (cont.)}
% Table created by stargazer v.5.1 by Marek Hlavac, Harvard University. E-mail: hlavac at fas.harvard.edu
% Date and time: Lun, Jun 15, 2015 - 10:38:08 p.m.
\begin{table}[!h] \centering
\scriptsize
\begin{tabular}{@{\extracolsep{0pt}}lcccc}
\\[-1.8ex]\hline
\hline \\[-1.8ex]
 & \multicolumn{4}{c}{\textit{Dependent variable: sh}} \\
\\[-1.8ex] & Inc & Ch & Inc (RS) & Cha (RS)\\
\hline \\[-1.8ex]
\\[-1.8ex] & (1) & (2) & (3) & (4)\\
\hline \\[-1.8ex]\\
 cprit & $-$0.06$^{*}$ & 0.04$^{***}$ & $-$0.07$^{*}$ & $-$0.01 \\
  & (0.03) & (0.01) & (0.04) & (0.04) \\
  comp & $-$0.01$^{***}$ & $-$0.002$^{***}$ & $-$0.01$^{***}$ & $-$0.001 \\
  & (0.002) & (0.0004) & (0.002) & (0.002) \\
  marginpre & $-$0.03 & 0.02 & 0.06 & 0.13$^{*}$ \\
  & (0.09) & (0.02) & (0.09) & (0.08) \\
  shpre2 & 0.45$^{***}$ & 0.25$^{***}$ & 0.34$^{***}$ & 0.67$^{***}$ \\
  & (0.08) & (0.03) & (0.08) & (0.07) \\
  Constant & 0.32$^{***}$ & 0.06$^{***}$ & 0.32$^{***}$ & 0.05 \\
  & (0.04) & (0.01) & (0.04) & (0.03) \\
 \hline \\[-1.8ex]
Observations & 123 & 265 & 80 & 112 \\
R$^{2}$ & 0.42 & 0.32 & 0.43 & 0.51 \\
Adjusted R$^{2}$ & 0.40 & 0.31 & 0.40 & 0.49 \\
F Statistic & 21.35$^{***}$ & 31.12$^{***}$ & 14.08$^{***}$ & 28.15$^{***}$ \\
\hline
\hline \\[-1.8ex]
\textit{Note:}  & \multicolumn{4}{r}{$^{*}$p$<$0.1; $^{**}$p$<$0.05; $^{***}$p$<$0.01} \\
\end{tabular}
\end{table}
\end{frame}


\begin{frame}\frametitle{Financiamiento de campaña en Argentina (cont.)}
\begin{itemize}\itemsep 15pt
\item Segmentando la muestra para incumbents y challengers, vemos que
  los resultados cambian significativamente.
\item $cprit$ signo positivo para challengers y signo negativo para
  incumbents.
\begin{itemize}
\item La unica explicación razonable que encontramos es que el signo
  negativo para incumbents refleje parte de un efecto sustitución
  asociado a tipos de financiamiento --i.e. formal vs informal.
\item Una forma de capturar esto sería tomar el gasto en publicidad
  oficial del incumbent en tiempos pre-electorales en determinados
  rubros --i.e. TV, cadena nacional, avisos FPT, etc. Aún así, no hay
 a priori una forma de asignar que fracción de ese gasto pueda
 asignarse a campaña.
\end{itemize}
\end{itemize}
\end{frame}


\begin{frame}\frametitle{¿Por qué donar a partidos políticos?}
\begin{itemize}\itemsep 15pt
\item Existen dos motivos principales asociados a la donación por
  parte de actores:
\begin{itemize}
\item Motivación ideológica $\longrightarrow$ actores contribuyen
  sobre bases ideológicas
\item Motivación lobby $\longrightarrow$ actores contribuyen sobre la
  base de beneficios esperados
\end{itemize}
\item En el segundo caso, los contribuyentes tienen incentivos a
  contribuir a mas de un partido --ie incumbent y principales
  challengers
\item ¿Cuánto contribuir? Grossman \& Helpman (1994, 1996) dicen hasta
  que el IM del ultimo peso invertido iguale al CM.
\end{itemize}
\end{frame}

\begin{frame}\frametitle{¿Por qué donar a partidos políticos?}
\begin{itemize}\itemsep 15pt
\item Existen varias maneras en que un representativo elegido puede
  ``devolver'' favores a sus contribuyentes.
\begin{itemize}
\item Politica comercial (demanda de proteccion)
\item Regulaciones bancarias/financieras
\item Licitaciones/contratos
\end{itemize}
\item Problema central $\longrightarrow$ commitment --no hay mecanismo
  de enforcement que obligue al gobierno.
\item Timing $\longrightarrow$ donante determina monto donación;
  gobierno implementa política; donante refuerza/retira apoyo.
\end{itemize}

\end{frame}


\begin{frame}\frametitle{¿Cuánto aportan los diferentes sectores a las
  campañas? }\vspace{-1cm}
  \begin{table}[htbp]
    \centering
 \caption{Mean contributions, by type of donor}
    \label{tab:1}
    \begin{tabular}[htbp]{|l|c|c|} \hline
      Tipo & Promedio & Nro aport\\ \hline \hline
Agro, ganaderia, caza & 9828 &  50      \\ \hline
Industria Alimentaria & 7222 & 24 \\ \hline
Industria Tabacalera & 18500 & 3 \\ \hline
Industria Madera, Papel, Impresiones & 12089 & 29 \\ \hline
Industria, Acero Metales y Herramientas & 5734 & 36 \\ \hline
Industria Automotriz y Transporte & 5373 & 15 \\ \hline
Industria Electrica y varios & 11375 & 16 \\ \hline
Transporte Energia y Gas & 19000 & 2 \\ \hline
Construccion y Edificacion & 5925 & 88 \\ \hline
Ventas al por mayor & 4197 & 92 \\ \hline
Servicios Transporte y Alm. Datos & 11600 & 32 \\ \hline \hline
    \end{tabular}
  \end{table}
\end{frame}


\begin{frame}\frametitle{Principales donantes (Argentina)}
\vspace{-1cm}
 \begin{table}[htbp]
    \centering
 \caption{Mean contributions, by type of donor}
    \label{tab:1}
    \begin{tabular}[htbp]{|l|c|c|c|} \hline
      Name &  Party & Type & Amount\\ \hline \hline
Asencio, Eduardo Marcelo & Concertacion UNA & Ind & 510000 \\ \hline
Sitrack.com Arg. SA & FPV & Emp & 400000 \\ \hline
Marsans Internacional SA & FPV & Emp & 400000 \\ \hline
CreaUrban SA & FPV & Emp & 400000 \\ \hline
ProIdeas SA & MPU & Emp & 400000 \\ \hline
La Inversora SA & MPU & Emp & 390000 \\ \hline
Encuentro para la Esperanza & Concertacion UNA & Emp & 390000 \\ \hline
Multipharma SA & FPV & Emp & 380000 \\ \hline
Pattriti SA & MPU & Emp & 380000\\ \hline
Iter Medicina SA & Emp & FPV & 360000 \\ \hline
Global Pharmacy SER SA & FPV & Emp & 310000 \\ \hline \hline
    \end{tabular}
  \end{table}

\end{frame}


\begin{frame}\frametitle{Grupos de intereses en Argentina}
\begin{figure}[htbp] \vspace{-0.25cm}
  \centering
  \includegraphics[scale=0.5]{figure4} \vspace{0cm}
  \caption{Contribuciones de empresas a partidos, 2003-2009}
  \label{fig:3}
\end{figure}
\end{frame}




\begin{frame}\frametitle{Grupos de intereses en Argentina (cont.)}
\begin{figure}[htbp] \vspace{0cm}
  \centering
  \includegraphics[scale=0.45]{grafo} \vspace{0cm}
  \caption{Audiencias con el presidente Macri (Fuente: elanalistaderedes.wordpress)}
  \label{fig:3}
\end{figure}
\end{frame}



\begin{frame}\frametitle{Grupos de intereses en Argentina (cont.)}
\begin{figure}[htbp] \vspace{-0.25cm}
  \centering
  \includegraphics[scale=0.5]{figure5} \vspace{0cm}
  \caption{Empresas y autoridades: conexiones}
  \label{fig:3}
\end{figure}
\end{frame}


\section{Lobby y conexiones políticas}

 \begin{frame}\frametitle{Motivation I - Little money}
    \begin{itemize}\itemsep 10pt
      \begin{block}{Tullock's puzzle: Why is there so little money in
          US politics?}
In 1972, total campaign spending in federal elections was
about \$200 million and total federal spending was \$400 bilion. In
2000, total campaign spending was around \$3 billion while total
federal government spending was around \$2 trillion.  The Federal government awarded \$134 billion in defense
contracts in 2000 while defense contracting firms and indviduals associated with
those firms donated only around \$13.2 million. In sum, value
of policy much larger than campaign contributions. 
\end{block}
\begin{block}{What about Argentina?}
In 2007, the national government awarded \$886 million in all public
contracts whereas the total amount of campaign contributions
\textit{by all firms and individuals} were
\$15 million. The figures for 2013 were \$1.62 billion and \$18 milion, respectively
  \end{block}
    \end{itemize}
  \end{frame}


\begin{frame}\frametitle{Motivation II - Informal connections}
  \begin{itemize}\itemsep 10pt
  \item How do IGs influence politics and policies
    in Argentina? 

   \begin{figure}[ht]
\centering
    \includegraphics[width=0.6\linewidth, height=0.15\textwidth]{mcdonalds}
  \caption{\small The day McDonalds Argentina ran out of ketchup}  \label{figure1}
\vspace{ -5 mm}
\end{figure}\vspace{-0.25cm}
\item Delays in DJAI; problem was solved within a day or two when the then chief of
  cabinet took notice of the situation.
    \end{itemize} \vspace{-0.25cm}
  \begin{block}{Channels of influence}
  ``In Argentina there are two ways to exert influence: one, more
  traditional, through formal meetings and opinion leaders, and
  another, directly related to electoral campaigns and the amount of
 \textit{under-the-table} contributions in exchange of future favors''
 [Unnamed lobbist source. Link:
 \href{https://www.lanacion.com.ar/1758062-lobby-empresario-el-poder-detras-del-poder}{La
   Nacion}]
  \end{block}
  \end{frame}



\begin{frame}\frametitle{Motivation III - Policy relevance}
  \begin{block}{Changes in political finance regime}
There were three major reforms to the regime of political finance in
Argentina in the last 15 years: the first was aimed at formalizing the
mixed system of political finance (2002); the second sought to
increase transparency and accountability (2007) and the third
prohibited contributions from firms and other legal persons (2009). Little to
no evidence as to how these changes impact on several outcomes. 
  \end{block}
  \begin{block}{Regulation of lobby activity}
There is currently a draft for a project bill regulating the activity
of interest groups. It is familiarly known as the project of ``Ley de
Lobby'' although the official project merely extends on the current
practice of recording hearings of interests. 
    \end{block}
\end{frame}



\begin{frame}\frametitle{Related literature}
  \begin{itemize}\itemsep 10pt
     \item Long-standing literature on campaign contributions and
       roll-call voting [Green and Krasno (1988), Palda and Palda (1998), Ansolabehere et al
  (2003), Stratmann (2005)]
        \item Electoral competition with special interest groups
  $\longrightarrow$ Baron (1994), Grossman \& Helpman (1992, 1996,
  2001)
\item Beyond campaign finance $\longrightarrow$ political connections
  and the revolving door [Vidal, Draca \& Fons-Rosen (2012), Acemoglu et al (2016)]
\item Timing of political influence  $\longrightarrow$ You (2014) states that around 50\% of lobbying activity
  in the US takes the form ex-post lobbying --i.e after Congressional
  vote.
  \begin{itemize}
  \item Our focus is on executive rather than legislative lobbying
    --extensive evidence that roll-call voting in Argentina is highly partisan
    [Jones (2001), Jones, Wang and Micozzi (2009)]. 
    \end{itemize}
    \end{itemize}
  \end{frame}



  \begin{frame}\frametitle{What we do}
  \begin{itemize}\itemsep 10pt
  \item Build a theoretical model where IGs decide on the optimal
    allocation between ex-ante and ex-post contributions aiming to
    obtain the highest share of public contracts awarded by the
    government [already]
    \item Test (partially) some of the theoretical implications
      regarding the size and direction of effects and the
      relationships between both channels of influence [in progress]
      \item Derive policy implications contributing to debate on
        institutional reform [in progress]
    \end{itemize}
  \end{frame}

   \begin{frame}\frametitle{Intuition and implications}
     \begin{itemize}\itemsep 10pt
       \item If IGs have similar preferences, ex-ante and ex-post
         contributions are perfect substitutes
         \item The higher the value of public contracts, the higher the
           incentive to engage in costly ex-post lobbying
           $\longrightarrow$ depends on value of contracts and amount
           of ex-ante contributions
  \item Also, there is likely
    to be room for cooperation rather than competitition between IGs
    (especially when it comes to ex-post lobbying) --i.e some evidence
    of UTE formation to apply for public tenders. 
    \item If IGs have extremely opposing preferences, then
      IGs have no incentive to make costly lobbying if they have
      contributed ex-ante; otherwise, they contribute a minimum amount
      of lobbying. 
       \end{itemize}
  \end{frame}


  \subsection{Model}

\begin{frame}\frametitle{Set up}
  \begin{itemize}\itemsep 10pt
  \item Election game between two candidates, $A$ and $B$
    \item $A$ wins the election with probability $P$; $B$ with $1-P$.
      \item Total expenditure on public contracts $V^k>0$ for the
        winning candidate $k$, where $k=A,B$
      \item Two interest groups, $i=1,2$. Can make:
        \begin{itemize}\itemsep 5pt \medskip
        \item Campaign contributions \textit{before} the election, $C_i>0$
        \item Lobbying contributions \textit{after} the election, $L_i>0$
        \end{itemize}
        \item IG's compete ex-post for the highest share, $\alpha^k$
          of the committed spending
        \end{itemize}
      
  \end{frame}


  
\begin{frame}\frametitle{Timing}
  \begin{figure}[ht]
\begin{center}
\textbf{Timing of the game}
\end{center}
\centering
    \includegraphics[width=0.75\linewidth, height=0.5\textwidth]{Timing}
  \caption{\small Time-structure of the model
  }
  \label{figure1}
\vspace{ 5 mm}
\end{figure}
  \end{frame}


  \begin{frame}\frametitle{Election outcome}
  \begin{itemize}\itemsep 10pt
  \item The probability of winning the election depends solely on the
    size and direction of the ex-ante campaign contributions $C_1$ and
    $C_2$ $\longrightarrow$ $P(C_1,C_2)$ probability that A wins the
    election. 
  \item There are two scenarios:
    \begin{itemize}\itemsep 5pt
    \item ``Aligned-preferences'' $\longrightarrow$ both IGs align with same candidate
      \item ``Opposite-preferences'' $\longrightarrow$ IGs support
        different candidates. 
      \end{itemize}
    \end{itemize}
      \begin{eqnarray}\label{u}
U_i = P(C_i, C_j) \left(\alpha^A V^A - L^A_i \right) + \left(1 - P(C_i, C_j)\right) \left( \alpha^B V^B - L^B_i  \right) - C_i   
\end{eqnarray}
  \begin{itemize}\itemsep 10pt
  \item where the entire campaign contribution, $C_i$ goes to the
    candidate who announces the highest spending, $V^k$. 
    \end{itemize}
  \end{frame}




  \begin{frame}\frametitle{Spending allocation rule}
    \begin{itemize}
    \item An IG's own contribution is a positive externality for it
      in the post-election period; and both contributions ($C_i$,
      $L_i$) are substitutes intertemporally. 
      \end{itemize}
    \begin{table}[ht]
\begin{center}
\textbf{Spending \textsl{V} allocation rule ($\alpha^k$)} \\
\vspace{3 mm}
\begin{tabular}[t]{l |c |c}
     &   $C_i = C^k_i$   &    $C_i \neq C^k_i$   \\
\hline
$C_j = C^k_j$ &   $\frac{L^k_i + C_i}{L^k_i + L^k_j + C_i + C_j}$  & $\frac{L^k_i }{L^k_i + L^k_j + C_j}$ \\
\hline
$C_j \neq C^k_j$ &   $\frac{L^k_i + C_i}{L^k_i + L^k_j + C_i}$  & $\frac{L^k_i }{L^k_i + L^k_j}$ \\
\end{tabular}
\caption{\scriptsize In columns: cases in which $i$'s supported candidate wins and loses the elections, $C_i = C^k_i$ and $C_i \neq C^k_i$ respectively. In rows, cases in which $j$'s supported candidate wins and loses the elections, $C_j = C^k_j$ and $C_j \neq C^k_j$ respectively. Each element in the matrix indicates the value of $\alpha^k$ from the combination of such scenarios.}
\end{center}
\end{table}
  \end{frame}


    \begin{frame}\frametitle{Case I: Aligned preferences (AP)}
  \begin{itemize}\itemsep 10pt
  \item If $V^A>V^B$, preferences of both IG are aligned. Any
    contribution go to candidate A.
  \item Ex-post problem $\longrightarrow$ how much lobby to exert
    after the election given a rival IG that also lobbies and given
    $C_i$ and $C_j$
    \item Ex-ante problem $\longrightarrow$ how much campaign
      contributions given the ex-post (lobbying) optimal behavior 
    \end{itemize}
  \end{frame}


    \begin{frame}\frametitle{Case I: Aligned preferences (cont.)}
\footnotesize
      \begin{block}{Proposition 1}
Ex-post lobby contributions are increasing in the total expenditure $V^k$, and if the ex-ante supported candidate:
\begin{itemize}
\item[(i)] takes office: ex-ante and ex-post contributions are perfect substitutes: $L^A_i + C^A_i = \frac{1}{4} V^A$, for: $i=1,2$ and $I=1$.  
\item[(ii)] does not takes office: $L^B_i = \frac{1}{4}  V^B$, for: $i=1,2$ and $I=0$.
\end{itemize}
In both cases shares $\alpha^A= \alpha^B=\frac{1}{2}$.
\end{block}
\begin{block}{Proposition 2}
For A, there exist $\check{V}^A < \hat{V}^A$ such that the $IGs$' ex-ante contributions exhibit an inverted U-shaped form with respect to $A$'s campaign expenditure $V^A$:
\begin{eqnarray} \nonumber
C^{*}_{i} = \left\{ \begin{array}{lll}
\vspace{2 mm}
\hspace{1 mm} C(V^A) & \hspace{2 mm} , \hspace{2 mm} if:\ V^A < \check{V}^A \\
\vspace{2 mm}
\frac{1}{2} - \frac{1}{4}  \left(  V^A -  V^B \right)  &  \hspace{2 mm} , \hspace{2 mm} if:\ V^A \in \left( \check{V}^A , \hat{V}^A  \right) & , \ for: \ i=1,2 \wedge i \neq j \\
\hspace{ 3 mm} 0 & \hspace{2 mm} , \hspace{2 mm} if:\ V^A > \hat{V}^A \\
 \end{array}
    \right.  
\end{eqnarray}
with $\frac{\partial C(V^A)}{\partial V^A} > 0$. 
\end{block}
  \end{frame}


    \begin{frame}\frametitle{Case I: Aligned preferences (cont.)}
\begin{figure}[ht]
\vspace{ 4 mm}
\begin{center}
\textbf{Reaction function for ex-post lobby contribution}
\end{center}
\begin{minipage}[b]{0.45\linewidth}
\includegraphics[width=1\linewidth, height=0.65\textwidth]{RF33}
\end{minipage}
\begin{minipage}[b]{0.45\linewidth}
\includegraphics[width=1\linewidth, height=0.65\textwidth]{RF444}
\end{minipage}
  \caption{\scriptsize $IG$ $i$'s lobby contribution $L^{k}_{i}$ in terms of the rival's lobby $L^{k}_{j}$ if the winner candidate was supported ex-ante (Left) and if it was not the case (Right). In both, the $i$'s optimal lobby response to $j$'s lobby behavior is to play aggressively each time that $j$ lobbies less than some threshold $\hat{L}^k_i$, and to `accommodate' the other way around. Threshold $\hat{L}^k_i$ crucially depends on the political outcome $V^k$ and also on the ex-ante contribution $C_j$ in the winning candidate was the supported ex-ante.}
\vspace{5 mm}
\end{figure}
  \end{frame}




   \begin{frame}\frametitle{Optimal behavior}
\begin{figure}[ht]
\vspace{ 4 mm}
\begin{center}
\textbf{Optimal contributive behavior}
\end{center}
\begin{minipage}[b]{0.45\linewidth}
\includegraphics[width=1\linewidth, height=0.65\textwidth]{optimo_A}
\end{minipage}
\begin{minipage}[b]{0.45\linewidth}
\includegraphics[width=1\linewidth, height=0.65\textwidth]{optimo_B}
\end{minipage}
  \caption{\scriptsize Optimal distribution of campaign and lobby contributions to the favorite candidate $A$ in terms of the announced payoff $V^A$ (Left), and optimal lobby contribution for the opposite candidate in terms of the announced payoff $V^B$ (Right).}
%\vspace{5 mm}
\end{figure}

  \end{frame}


    \begin{frame}\frametitle{Case I: Aligned preferences (cont.)}
  \begin{itemize}\itemsep 10pt
  \item Campaign contributions, $C_i$ are a useful instrument to bias
    the likelihood of winning for a given candidate; lobbying
    contributions (activities), $L_i$ are (almost) a total waste of
    resources --i.e $\alpha^k$ do not depen on these. 
    \item IGs may find it optimal to coordinate their strategies
      --especially, lobbying. They can improve their results by
      reducing their lobbying contributions to some minimum.
          \end{itemize}
  \end{frame}


 %    \begin{frame}\frametitle{Case II: Opposite preferences}
%   \begin{itemize}\itemsep 10pt
%   \item Campaign contributions in opposite directions
%     $\longrightarrow$ $i$ donates to $A$, $j$ donates to $B$.
%     \item Only way to increase the likelihood of winning of favourite
%       candidate for an IG is to contribute more than the other.
%       \item We assume that the IG derive utility only from a single
%         candidate (extreme-opposite preferences). 
%     \end{itemize}
%   \end{frame}


%     \begin{frame}\frametitle{}
%  \begin{block}{Proposition 3}
% In the game with extreme opposite preferences, political contributions are allocated to a single objective, either the campaign or lobbying, according to the following rule:

% Given $i$'s favorite candidate - candidate $A$ - taking office:
% \begin{itemize}
% \item[(i)] if $C_i>0$: there is no ex-post lobbying and ex-post utilities are given by $\left( U^{EP}_i , U^{EP}_j \right)= \left( V^A  , 0 \right)$,
% \item[(ii)] if $C_i=0$: the $IG$ $i$ is the only one that lobbies, particularly $L^A_i= \epsilon>0$, $\epsilon \rightarrow 0$, and ex-post utilities are given by $\left( U^{EP}_i , U^{EP}_j \right)= \left( V^A-\epsilon ,  0 \right)$.
% \end{itemize}
% The analogous rule holds in the case in which $j$'s favorite candidate ($B$) takes office.%, but in there $j$ is the $IG$ that achieves a positive payoff.
% \end{block}
%   \end{frame}


  \subsection{Data and methodology}
  
 \begin{frame}\frametitle{Data: Main databases}
\begin{itemize}\itemsep 15pt
\item Data compiled from several sources leads to three main datasets
  all at the individual level of observation
  \begin{itemize}\itemsep 5pt \medskip
    \item Data on campaign contributions $\longrightarrow$ over 46000
  individual-level campaign contributions for National elections from
  2003 to 2015 [Sources: Cámára Nacional Electoral, Poder Ciudadano's
  Dinero y Política project and la Ruta Electoral project]
  \item Data on hearings of interest (``audiencias de interés'')
    $\longrightarrow$ nearly 70000 records of recorded hearings
    between members of the executive, the cabinet and directors and
    individuals representing themselves or an organized interest
    during 2003-2015
    [Source: Registro Nacional de Audiencias de Interés]
\item Data on public procurement contracts $\longrightarrow$ including
  individual and firms participating in public procurement contracts
  comprising purchases of goods and services and public works [Source:
  Boletin Oficial Nacional]
  \end{itemize}
\end{itemize}
\end{frame}

\begin{frame}\frametitle{Data: Dictionary databases}
  \begin{itemize}\itemsep 15pt
  \item AFIP adminstrative records (padrón de contribuyentes)
    $\longrightarrow$ over 4.6 million records containing names and
    IDs (CUIT number) for both natural and legal persons.    activity codes for 480000 legal entities
    \item Registered legal entities and
      authorities (Inspección General de Justicia) $\longrightarrow$
      over 1.2 million records containing CUIT and membership type
      (partner, director, etc)
      \item Sistema Integrado Previsional Argentino (SIPA)
        $\longrightarrow$ administrative records on size and type of
        firms. 
\end{itemize}
  \end{frame}

  
    \begin{frame}\frametitle{Matching IDs}
  \begin{itemize}\itemsep 10pt
  \item Common unique identifier in all databases $\longrightarrow$
    CUIT number.
    \begin{itemize}
    \item Many missing data on CUIT
      --this is mostly a problem for both public contract and hearings
      of interest data. 
      \item The raw data on public contracts scrapped from the Boletín
        Oficial contained only about 15\% of CUIT data. Since public
        contracts can be awarded to both natural and legal persons,
        one cannot know exactly who the contract was awarded to!
      \item When ``cuit'' is non-missing, matching is straightforward
        --exact-matching on ``cuit''. 
      \item When ``cuit'' is missing, we followed two procedures:
        \begin{itemize}
        \item Manual recovery of ``cuit'' data $\longrightarrow$
          \item String-matching (matching based on ``name'') against a
            dictionary database.
            \begin{itemize}
            \item Exact-string matching
              \item Fuzzy-string matching algorithm based on ``optimal
                string alignment''
              \end{itemize}
              \end{itemize}
      \end{itemize}
    \end{itemize}
\end{frame}

      \begin{frame}\frametitle{Some descriptive data}
  \begin{table}[!htbp]
  \centering
 \caption[Private and public campaign contributions]{Private and
    public campaign contributions, 2005-2015 (All parties)}
  \label{tab:pripub}
\begin{footnotesize}
  \begin{tabular}[!htbp]{lcccccc}
Concept	&	2005	&	2007	&	2009	&	2011	&	2013	&	2015	\\\hline\hline
Pri (mill 2015 pesos )	&	77.21	&	193.54	&	301.63	&	95.00	&	242.08	&	226.37	\\
Pub (mill 2015 pesos)	&	55.88	&	137.14	&	113.85	&	637.59	&	252.35	&	747.53	\\
Total (mill 2015 pesos)	&	133.09	&	330.68	&	415.48	&	732.59	&	494.43	&	973.90	\\
Pri (\%)	&	58.01	&	58.53	&	72.60	&	12.97	&	48.96	&	23.24	\\
Pub (\%)	&	41.99	&	41.47	&	27.40	&	87.03	&	51.04	&	76.76	\\
Total (\%)	&	100	&	100	&	100	&	100	&	100	&	100	\\
  \end{tabular}
  \end{footnotesize}
   \end{table}
\end{frame}
  

\begin{frame}\frametitle{Some descriptive data (cont.)}
\begin{table}[!htbp]
  \centering
 \caption{Distribution of public contracts, by year}
  \label{tab:contracts}
\begin{footnotesize}
  \begin{tabular}[!htbp]{lccc}
    Year	&	# Contracts	&	Total amount (mill $)	&	Avg amount (mill $)	\\\hline\hline
2003	&	954	&	1112.25	&	1.17	\\
2004	&	1914	&	5840.57	&	3.05	\\
2005	&	2385	&	7732.18	&	3.24	\\
2006	&	2544	&	7953.07	&	3.13	\\
2007	&	2722	&	11519.81	&	4.23	\\
2008	&	3972	&	13631.77	&	3.43	\\
2009	&	4368	&	18509.29	&	4.24	\\
2010	&	3871	&	14626.33	&	3.78	\\
2011	&	4891	&	16622.97	&	3.40	\\
2012	&	4866	&	23759.95	&	4.88	\\
2013	&	3267	&	24559.17	&	7.52	\\
2014	&	2582	&	16213.58	&	6.28	\\
2015	&	3264	&	9106.31	&	2.79	\\\hline\hline
Total	&	41600	&	171187.24	&		
  \end{tabular}
  \end{footnotesize}
\end{table}
\end{frame}

      \begin{frame}\frametitle{Some descriptive data (cont.)}
\begin{table}[!htbp]
  \centering
  \caption[Distribution of # of publc tender contracts]{Distribution
    of public tender contracts - By firm/person and # of contracts
    awarded}
  \label{tab:lic1}
  \begin{tabular}[!htbp]{lcc}
    # of contracts	&	# firms/persons	&	\% 	\\\hline\hline
1 a 5	&	11409	&	0.838650397	\\
6 a 20	&	1552	&	0.114084093	\\
21 a 50	&	450	&	0.033078506	\\
51 a 100	&	132	&	0.009703029	\\
more than 100	&	61	&	0.004483975	\\\hline
Total	&	13604	&	1	\\
  \end{tabular}
\end{table}
\end{frame}



      \begin{frame}\frametitle{Some descriptive data (cont.)}
\begin{figure}[!htbp]
  \centering
 \caption{Public tender contracts awarded (up to
    \$1 million 2015 pesos)}
  \label{fig:lic1}
  \includegraphics[scale=0.075]{lic1}
 \end{figure}
\end{frame}



\subsection{Strategy}
      \begin{frame}\frametitle{Empirical strategy}
  \begin{itemize}\itemsep 10pt
  \item We are to perform two types of analysis. 
  \end{itemize}
  \begin{equation}
  Y_{i}=\sum_{h=0}^{H}y_{i,t+h}=\alpha+\beta C_{i,E}+\gamma
  \sum_{h=0}^{H} \omega_{t+h}L_{i,t+h}+\epsilon_{i}
\end{equation}
\begin{itemize}
\item we use only information on actors who obtained a positive
amount of public procurement contracts, so that $\beta$
and $\gamma$ will reflect just the existence of correlations with the
dependent variable; we expect both coefficients to be positive.
  \end{itemize}
\end{frame}

  \begin{frame}\frametitle{Empirical strategy (cont.)}
  \begin{itemize}\itemsep 10pt
  \item For the subsample in which we have both winners and
losers of a given public procurement contract, specification (1) can
be estimated with variable Y defined in a way that it takes only two
possible values, 1 when an interested actor was granted at least one
procurement contract, and 0 otherwise.
  \end{itemize}
  \begin{equation}
P_{i,j}=\alpha+\beta f(C_{i \in j,E})+\gamma h(L_{i \in j,t+h})+\epsilon_{i}
\end{equation}
\begin{itemize}
\item where $j$ identifies a specific bidding process; $f(.)$ and
$h(.)$ are functions modelling the relationships between campaign
contributions and lobby efforts among all interested actors that
participated of the bidding process $j (i \in j)$. The variable $P_{i,j}$ takes the value 1 if interested actor $i$ won the bidding process and 0 otherwise\
  \end{itemize}
\end{frame}
  

\end{document}

\section{Corrupción}



\begin{frame}\frametitle{La corrupción como problema}
\begin{list}{}
\item \textbf{``Hemos identificado a la corrupción como el principal obstáculo al
desarrollo económico y social''} [Banco Mundial] \bigskip
\item \textbf{"La corrupción ha dejado de ser un problema local para convertirse en
un fenómeno transnacional que afecta a todas las sociedades y
economías, lo que hace esencial la cooperación internacional para
prevenirla y luchar contra ella"} [Convención de las Naciones Unidas
contra la Corrupción] \bigskip 
%\item ``Corruption is one of the greatest challenges of the
%contemporary world. It undermines good government, fundamentally distorts public
%policy, leads to the misallocation of resources, harms the private sector and private
%sector development and particularly hurts the poor.''
\item \textbf{"La corrupción atrapa a millones en la pobreza"} [Transparencia
  Internacional] \bigskip 
\end{list}
\end{frame}


\begin{frame}\frametitle{¿De qué hablamos cuando hablamos de corrupción?}
\begin{itemize}\itemsep 10pt
\item Enfoque moderno $\longrightarrow$ se define a la corrupción como
  el uso ilegítimo del poder público para el beneficio privado. 
\begin{itemize}
\item Importa un acto ilegítimo e ilegal 
\item Involucra una posición de poder público
\item Se persigue un beneficio de apropiación individual
\end{itemize}
\item El estudio de la corrupción definida de esta manera se
  desentiende de temas como la corrupción entre privados
  --i.e. departamento de compras y ventas de dos empresas; entre
  individuos; mafias.  
\end{itemize}
\end{frame}


\begin{frame}\frametitle{El abordaje económico de la corrupción}
\begin{itemize}\itemsep 10pt
\item En disciplinas como antropología cultural o sociología, el
  abordaje del fenómeno enfatiza normas sociales y valores morales;
  mayores niveles de corrupción se toman como un signo de degradación
  moral.
\item La economía enfatiza dos elementos: a) incentivos; y b)
  organizaciones. 
\begin{itemize}
\item Las normas sociales varían ampliamente entre países
  --i.e. freebies a políticos y burócratas; lealtad al clan y al
  linaje preceden a la ocupación política. 
\end{itemize}
\item Explicaciones culturales de la corrupción son importantes;
pero, riesgo de explicación tautológica. 
\end{itemize}
\end{frame}


\begin{frame}\frametitle{El abordaje económico de la corrupción (cont.)}
\begin{itemize}\itemsep 10pt
\item El abordaje económico de la corrupción busca explicar porque
  países (regiones) similares llegan a diferentes estándares de normas
  sociales estables en el tiempo (equilibrios). Adicionalmente, es relevante
  estudiar cómo los países pueden trasladarse de un equilibrio a
  otro. 
\begin{itemize}
\item Esta dinámica de cambio puede ocurrir de manera muy lenta
  --Africa subsahariana- o relativamente rápido --caso de
  Singapur. Elemento clave: impacto de los cambios en las políticas
  sobre la estructura de incentivos. 
\end{itemize}
\item  PONER
\end{itemize}
\end{frame}


\begin{frame}\frametitle{Un ejemplo}
\vspace{-12pt}
\begin{scriptsize}
\begin{block}{}
\texttt{Me cuesta
levantarme en diciembre: mi cuerpo me dice que debería estar
hibernando en lugar de ducharme, afeitarme y conduciendo a través de
los terribles embotellamientos de Moscú. Así que estoy siempre \underline{a punto
de llegar tarde}, y no me gusta esa sensación. Ergo, \underline{me tomé un atajo} y
un policía tráfico me hizo detener. No hacía falta que dijera nada,
pero lo hizo.} \bigskip 

\texttt{``Esa no es manera de conducir'', dijo} \bigskip 

\texttt{Yo tampoco tenía que decir nada y no lo hice. Sólo le entregué mi
licencia de conducir \underline{envuelta en un billete de 500 rublos}. El policía
me devolvió la licencia y me saludó. Yo seguí mi camino. Ahora \underline{no tendría que ir al banco a pagar la multa}. Pero
seguramente \underline{iba a llegar tarde}: al hacerme detener de manera algo
innecesaria, el policía me había hecho perder tres de mis \underline{preciados}
minutos de la mañana.}  \medskip 

[Leonid Bershidsky, escritor y periodista] 
\end{block}
\end{scriptsize}
\end{frame}


\subsection{Teoría}

\begin{frame}\frametitle{La corrupción como un ``contrato''}
\begin{block}{Configuración de la situación de corrupción}
El sector público está siempre involucrado -a través el Estado,
burócratas, funcionarios. Es una de las partes del contrato, es la demanda. La otra
parte generalmente proviene de la sociedad civil --invididuos,
empresas, sindicatos. Es la oferta. Se configura así un contrato
secreto de intercambio entre las dos partes. El hecho generador es una
posición de poder asimétrico que permite a una de las partes exigir
algo a cambio de un bien/servicio que debería brindar en cumplimiento
de su función.  
\end{block}
\end{frame}





\begin{frame}
  \begin{figure}[htbp]
    \centering
    \includegraphics[scale=0.7]{acuerdocorrup}
    \caption{El proceso de la corrupción como un ``contrato''}
    \label{fig:corrupcioncpi}
  \end{figure}
\end{frame}

\begin{frame}\frametitle{Iniciación del acuerdo}
\begin{itemize}\itemsep 10pt
\item El primer paso es
encontrar un socio y, luego, negociar el contrato. En el mundo de la corrupción,
¿cómo y dónde encontrar el socio adecuado? El socio debe tener
capacidad, voluntad y \textit{savoir-faire} $\longrightarrow$
``signaling''--i.e. fama, intermediarios, conexiones legales
(conocidos, familiares, socios, etc.).  
\item  Luego se debe negociar el acuerdo: ¿quién, cómo y cuánto?. Tema
  del \textit{quid-pro-quo}. ¿Hay estándares? $\longrightarrow$ riesgo
  alto.
\item Ocasionalmente, la estrategia suele ser ``freebies'' o
  puestos/cargos en lugar de dinero. Más sutil pero más imprecisa. 
\end{itemize}
\end{frame}


\begin{frame}\frametitle{Ejecución del acuerdo}
\begin{itemize}\itemsep 10pt
\item Incentivos para no cumplir; mecanismos para forzar cumplimiento
  y costos de no cumplir $\longrightarrow$ rehenes, contratos legales
  e intermediarios, violencia (amenazas), integración vertical,
  reputación, relaciones y aspectos sociales. 
\begin{itemize}\itemsep 5pt
\item ``Rehenes'' $\longrightarrow$ pagos a cuenta, señas; no se
  elimina el problema
\item Apariencia legal $\longrightarrow$ pago de
  comisiones por intermediación. Alto riesgo
\item Violencia y extorsión, creíble y efectiva
\item Integración vertical de ambas partes corruptas (joint venture). 
\item Reputación y repetición $\longrightarrow$ posibilidad de
  sanciones. 
\end{itemize}
\end{itemize}
\end{frame}


\begin{frame}\frametitle{Post-acuerdo}
\begin{itemize}\itemsep 10pt
\item Debido a la ilegalidad, un contrato corrupto no tiene fecha de
  finalización; interdependencia mutua. Incentivos latentes a
  ``soplar''. Puede funcionar como amenaza o como acción directa. 
\item ``Reciprocidad negativa'' $\longrightarrow$ causar daño a la
  parte que me engañó --a pesar de consecuencias negativas para mi
  persona. 
\item Penas y castigos asimétricos $\longrightarrow$ funcionario mas
  expuesto; situación de poder asimétrico favorable a la parte civil. 
\item Caso: Bochum (ALE) un funcionario informó los nombres de los
  competidores en una licitación por autopistas; recibió luego un
  sobre con 2000 euros de la empresa adjudicataria para comprar el
  silencio.  
\end{itemize}
\end{frame}




\begin{frame}\frametitle{Corrupción: algunos hechos estilizados}
\begin{itemize}\itemsep 10pt
\item El Banco Mundial estima el costo directo anual de coimas y sobornos en más de un
trillón de dólares
\item La corrupción ha fomentado la fuga de capitales en Africa por
  más de $U\$S$ 400 billones; un cuarto de eso corresponde a Nigeria.
\item En Mexico, el hermano de Carlos Salinas atesoró más de $U\$S$
  120 millones, lo que equivaldría a cubrir los costos de salud
  anuales para casi 600 mil personas. 
\item Invertir en un país corrupto puede ser hasta un 20\% más caro
  que en un país no corrupto. 
\end{itemize}
\end{frame} 


\begin{frame}\frametitle{Corrupción: algunos hechos estilizados (cont.)}
\begin{itemize}\itemsep 10pt

\item La corrupción no es un fenómeno exclusivo de regímenes
  autocráticos o países sub-desarrollados: en países democráticos
  --Venezuela, Paraguay, Filipinas-- y países desarrollados --Italia,
  República Checa- persisten altos niveles de corrupción 
\item Experiencias de corrupción muy variables entre países
  --i.e. Bangladesh/Corea del Norte vs Finlandia;
las percepciones acerca de este fenómeno también lo son
\item La corrupción tiene ciertas características endémicas y
  características de correlación autoespacial --Africa sub-sahariana;
  sudeste asiático; américa latina. 
\item La corrupción tiene costos asociados que son cuantificables (estimables)
\end{itemize}
\end{frame}

\begin{frame}
 \begin{figure}[htbp]
    \centering
    \includegraphics[scale=0.09]{TI_CPI_2011_infographic}
    \caption{Incidencia de la corrupción por regiones 2011}
    \label{fig:corrupcionlarge}
  \end{figure}
\end{frame}


\begin{frame}\frametitle{Tipos de corrupción pública}
Se pueden distinguir diferentes tipos de corrupción según varios
criterios:
\begin{itemize}\itemsep 10pt
\item Corrupción burocrática vs corrupción política
\item Corrupción extorsiva vs corrupción colusiva
\item Corrupción centralizada vs corrupción descentralizada
\end{itemize}
\end{frame}


\begin{frame}\frametitle{Corrupción burocrática y política}
\begin{itemize}\itemsep 10pt
\item La distinción más usada es que refiere a la:
\begin{itemize}\itemsep 10pt
\item corrupción burocrática (petty corruption) $\longrightarrow$ involucra el mal uso de
  cargo público para un beneficio privado y está vinculada con agentes
  públicos de cualquier esfera y jurisdicción --i.e. coimas a agentes
  de tránsito y control. 
\item corrupción política (grand corruption) $\longrightarrow$ involucra el mal uso del
  cargo público y el poder político para un beneficio privado y está
  vinculada con la maxima jerarquía política y funcionarios
  designados --i.e. coimas en el Senado. 
\end{itemize}
\item Distinción no siempre útil y apropiada en todos los contextos. 
\end{itemize}
\end{frame}


\begin{frame}\frametitle{Corrupción burocrática y política (cont.)}
\begin{itemize}\itemsep 10pt
\item En regímenes comunistas --incluido China, Vietnam- y
  autoritarios, la distinción es prácticamente trivial. Algo similar
  pasa en países democráticos en que los burocrátas de máxima
  jerarquía son designados políticamente y no agentes públicos de
  carrera. 
\item En muchos casos la corrupción burocrática (administrativa)
  difiere de la política: países democráticos con
  alta competencia electoral $\longrightarrow$ políticos mucho
  más expuestos que los burócratas. 
\item Otros países tienen una muy baja incidencia de
  corrupción burocrática pero muy alta corrupción
  política (EEUU; lobbies; laws for sale). Otros países tienen
  alta incidencia de ambos tipos (India, Argentina) aunque la
  corrupción burocrática es muy extendida.  
\end{itemize}
\end{frame}


\begin{frame}
\begin{figure}[htbp]
    \centering
    \includegraphics[scale=0.25]{GCB201011_Infographic}
    \caption{¿Qué países perciben sus instituciones como corruptas?}
    \label{fig:corruptinstit}
  \end{figure}
\end{frame}


\begin{frame}\frametitle{Corrupción burocrática y política (cont.)}
\begin{itemize}\itemsep 10pt
\item Una distinción importante entre corrupción política y corrupción
  burocrática es en relación a la oportunidad de cada una. La
  corrupción política al involucrar decisores de política, tiene lugar
  \textit{al nivel del diseño de la política.}
\item La corrupción burocrática en cambio, al involucrar burócratas de
  todo tipo y nivel generalmente tiene lugar \textit{al nivel de la
  implementación de la política.} 
\item Esta distinción es importante por cuanto ambas tienen efectos
  diferentes. La corrupción política afecta la manera en que son
  tomadas las decisiones. En particular, se manipulan las
  instituciones políticas y las reglas de modo que afecta su
  funcionamiento efectivo. 
\end{itemize}
\end{frame}


\begin{frame}\frametitle{Corrupción burocrática y política (cont.)}
\begin{itemize}\itemsep 10pt
\item La corrupción política está casi siempre presente en los
  regímenes autoritarios. De hecho, la corrupción política es
  frecuentemente la base de la acumulación de poder, es inherente a lo
  lógica del sistema. 
\item Sin embargo la corrupción política no es
  exclusiva de regímenes autoritarios. Caso de EEUU y lobbies; la ley
  Banelco en Argentina; \textit{mensalao} en Brasil, etc. 
\item Pero a diferencia de los regímenes autoritarios, la corrupción
  política tiene rasgos más bien episódicos. 
\end{itemize}
\end{frame}



\begin{frame}\frametitle{Corrupción extorsiva y colusiva}
\begin{itemize}\itemsep 10pt
\item corrupción extorsiva $\longrightarrow$ aquella cuando una de las partes exige a la otra el pago
de una suma extra (sobreprecio, coima, soborno) para proveerle un bien
o servicio. Los burócratas reciben sobornos para hacer lo que ellos deben
  hacer. Ejemplos.
\item corrupción colusiva $\longrightarrow$ aquella cuando una parte
  acuerda con la otra una transacción de modo que produzca beneficios
  a ambas partes. Es cuando un burócrata es coimeado para hacer lo que
  no se supone que haga. Ejemplos. 
\item En este caso, ambos tipos de corrupción tienen claras
  implicancias diferentes
\end{itemize}
\end{frame}



\begin{frame}\frametitle{Corrupción extorsiva y colusiva (cont.)}
\begin{itemize}\itemsep 10pt
\item En el primer caso, los objetivos de ambas partes pueden estar en claro
  contraste; se paga para acelerar un trámite pero nada garantiza un
  resultado exitoso ni que no se deban pagar sobornos adicionales
  --incentivos perversos; contrato implícito not-enforceable.
\item En el segundo caso, los objetivos de ambas partes están
  alineados. No existen incentivos (en principio) para que alguna de
  las partes vuelva sobre sus pasos o denuncie el acto. 
\item La incidencia de este tipo de corrupción suele resultar en mayor
  persistencia de prácticas corruptas y de mayores consecuencias
  negativas asociadas a la corrupcion. 
\end{itemize}
\end{frame}


\begin{frame}\frametitle{Analisis micro}
\begin{itemize}\itemsep 10pt
\item Suponga un gobierno que provee un bien (pasaporte,
  licencia). Bien homogéneo. Existe una demanda privada por este bien
  $\longrightarrow$ $D(p)$. Es vendido por un agente que \textit{puede
    efectivamente restringir la cantidad que se vende}
\item Puede hacer esto sin ningún riesgo de ser detectado y
  castigado. Esto lo hace un monopolista que vende un bien. 
\item Suponemos que el objetivo del agente es maximizar la cantidad de
  coimas que recibe de la venta de ese bien.
\end{itemize}
\end{frame}



\begin{frame}\frametitle{Analisis micro (cont.)}
\begin{itemize}\itemsep 10pt
\item Supongamos el precio oficial del bien es $p$. Suponemos además
  que el agente no tienen ningún costo propio en la provisión del
  bien. 
\item ¿Cuál es el costo marginal de proveer este bien? Distinguimos
  dos casos. 
\begin{itemize}\itemsep 15pt \medskip 
\item Caso sin robo $\longrightarrow$ el agente integra el precio al
  gobierno y además cobra coima. El costo marginal es $p$. 
\item Caso con robo $\longrightarrow$ el agente no integra el precio
  al gobierno (oculta la transacción) y cobra coima. El costo marginal
  es $0$. 
\end{itemize}
\item Como monopolista fijara la venta en donde IM=CM. En el caso de
  corrupción sin robo restringira cantidad; en corrupcíon con robo
  no. 
\end{itemize}
\end{frame}


\begin{frame}\frametitle{Analisis micro (cont.)}
  \begin{figure}[htbp]
    \centering \vspace{-2cm}
    \includegraphics[scale=0.4]{shlei1}
    \caption{Corrupción sin robo}
  \end{figure}
\end{frame}


\begin{frame}\frametitle{Analisis micro (cont.)}
  \begin{figure}[htbp]
    \centering \vspace{-2cm}
    \includegraphics[scale=0.4]{shlei2}
    \caption{Corrupción sin robo}
  \end{figure}
\end{frame}


\begin{frame}\frametitle{Analisis micro (cont.)}
\begin{itemize}\itemsep 10pt
\item Los dos casos son conceptualmente similares. 
\end{itemize}
\end{frame}


\begin{frame}\frametitle{Corrupción centralizada y descentralizada}
\begin{block}{}
Cuando cayó el régimen extremadamente corrupto de Suharto en Indonesia
se pensaba que podría haber cambios favorables en esta
problemática. Sin embargo, la situación ha resultado ser peor que
cuando estaba la familia Suharto. 
\end{block}
\begin{itemize}\itemsep 10pt
\item Tema central $\longrightarrow$ como está organizada
  ``industrialmente'' la corrupción. Puntos de coima; actores;
  coordinación; \textit{one-stop-corruption-shop}.
\end{itemize}
\begin{block}{}
Algo similar pasó en la Rusia post-Soviética, en donde se originaron
múltiples instancias de demoras administrativas que fomentaron un tipo
de corrupción mucho más anárquica y desorganizada. 
\end{block}
\end{frame}


\begin{frame}\frametitle{Corrupción centralizada y descentralizada (cont.)}
\begin{itemize}\itemsep 10pt
\item Esta cuestión es particularmente interesante ya que hay
  experiencias de países que han tenido un alto crecimiento económico
  a la vez que altísimos niveles de corrupción --Corea. Algunas
  explicaciones sugieren que el gobernante pedía pagos de suma fija a
  todos los conglomerados \textit{independientemente} de sector de
  actividad y de las políticas de gobierno. Este tipo de corrupción
  suele ser menos distorsiva en términos de mala asignación de los
  recursos. Caso Bangladesh, ultima decada. 
\item Esta puede ser una de las razones por
  las que las experiencias de China y Rusia difieren: mientras ambos
  países han descentralizado la actividad económico, China tiene aún
  un fuerte centralismo político; Rusia está políticamente
  descentralizada.  
\end{itemize}
\end{frame}



\subsection{Medición}



\begin{frame}\frametitle{La medición de la corrupción}
\begin{itemize}\itemsep 10pt
\item El fenómeno de la corrupción es difícil de identificar, y mucho
  más difícil aún de medir en términos cuantitativos. No obstante, en
  los últimos treinta años se diseñaron varios indicadores --CPI, WBC,
  Berterlsmann Index, ICRG Corruption etc.
\item La mayoría comparten una característica $\longrightarrow$ son
  mediciones acerca de la percepción de corrupción. 
\item Existen naturalmente varios problemas con este tipo de mediciones.
\end{itemize}
\end{frame}


\begin{frame}\frametitle{La medición de la corrupción (cont.)}
\begin{itemize}\itemsep 10pt
\item La percepción está influida por la experiencia diaria del país
  donde se toma --ej India vs EEUU
\item Se basan en las percepciones de empresarios, inversores y otros
  grupos extranjeros --i.e. puede diferir bastante de la experiencia
  del empresariado local
\item La gente de negocios generalmente asigna mejores ``notas'' a
  países con mejor performance económica --caso China, alta corrupción
  y alto crecimiento. 
\item Estas mediciones suelen estar limitadas al aspecto de la
  corrupción burocrática y no de la corrupción política. 
\end{itemize}
\end{frame}



\begin{frame}\frametitle{La medición de la corrupción (cont.)}
\begin{itemize}\itemsep 10pt
% \item A pesar de estos problemas, estos indicadores representan una
%   primera aproximación a dimensionar el problema en los diferentes
%   países. 
\item Medidas objetivas $\longrightarrow$ cantidad de funcionarios
  encarcelados; juicios iniciados; sentencias dictadas; menciones en
  diarios. 
\begin{itemize}
\item Problema $\longrightarrow$ corrupción es secreta y clandestina,
  virtualmente imposible de medir objetivamente su dimensión. 
\end{itemize}
\item Una posibilidad $\longrightarrow$ programa de monitoreo de proyectos de
  construcción de rutas en Indonesia [Olken (2005)]; comparación de
  materiales gastados con materiales efectivamente usados.
\item Otra posibilidad: tomar características institucionales
  --prácticas licitatorias, procedimientos presupuestarios. Problema:
  ``reglas del juego'' vs ``juego del juego''. 
\end{itemize}
\end{frame}


\begin{frame}\frametitle{La medición de la corrupción (cont.)}
\begin{itemize}\itemsep 10pt
% \item A pesar de estos problemas, estos indicadores representan una
%   primera aproximación a dimensionar el problema en los diferentes
%   países. 
\item Existen varios mitos acerca del tema de la medición de la
  corrupción. Algunos de los más relevantes son:
\begin{itemize}\itemsep 10pt
\item que la corrupción no puede ser medida. 
\item que las percepciones son vagas y no reflejan la realidad
\item mediciones subjetivas no confiables 
\item mediciones subjetivas no son ``accionables''
\item se necesitan medidas objetivas
\end{itemize}
\item En última instancia, la lucha efectiva contra la corrupción
  requiere no sólo de voluntad política y de un esfuerzo integral sino
  también de un correcto diagnóstico de la dimensión y características
  del problema. 
\end{itemize}
\end{frame}


\begin{frame}\frametitle{¿Efectos positivos de la corrupción?}
Hace varias década algunos autores [Leff, Lui] sugirieron que podían
existir efectos positivos asociados a prácticas de corrupción \smallskip

\begin{block}{}
Corruption may introduce an element of competition into what is otherwise a comfortably monopolistic
industry....[and] payment of the highest bribes [becomes] one of the principal criteria for
allocation....Hence, a tendency toward efficiency is introduced into the system.
\end{block}
\smallskip 
\begin{block}{}
Bribing strategies...minimize the average value of the time costs of the queue....[and the
official]...could choose to speed up the service when bribery is allowed."
\end{block}
\end{frame}


\begin{frame}\frametitle{¿Efectos positivos de la corrupción? (cont.)}
\begin{itemize}\itemsep 10pt
\item El argumento central es que el pago de sobornos puede ser un
  mecanismo eficiente para pasar por alto pesadas regulaciones y
  legislación inefectiva. Muchas empresas están dispuestas a hacerlo
  de esta manera siempre que los beneficios derivados de esquivar la
  regulación sean mayores a los costos de hacerlo. 
\item Pero el argumento tiene muchos baches:
\begin{itemize}\itemsep 10pt
\item Ignora el nivel de discrecionalidad que tienen los políticos y
  burócratas especialmente en sociedades con alta incidencia de
  corrupción $\longrightarrow$ problema mas grave
\item El argumento de ``speed money'' presupone que ambas partes van a
  cumplir con lo acordado y no tendrán incentivos a pedir sobornos
  adicionales. 
\end{itemize}
\end{itemize}
\end{frame}

\begin{frame}\frametitle{¿Efectos positivos de la corrupción? (cont.)}
\begin{itemize}\itemsep 10pt
\item Una versión moderna dice que la corrupción permite
  la operación del mecanismo de O y D. Ejemplo: En una licitación
  pública donde todos tengan la misma oferta terminará ganando el que
  pague el soborno más alto --es decir, la empresa más eficiente. 
\item Interesante pero problema $\longrightarrow$ ignora que la
  corrupción es un robo de recursos públicos --las coimas no van a los
  recursos del tesoro y terminarán en el circuito informal/banca
  offshore
\item Otro problema es que supone mayor coima-mayor eficiencia;
  contra-argumento es que la mayor coima se financia con menor
  calidad (ej. infraestructura, caminos, etc.). 
\end{itemize}
\end{frame}


\begin{frame}\frametitle{¿Efectos positivos de la corrupción? (cont.)}
\begin{itemize}\itemsep 10pt
\item En definitiva, este enfoque de la corrupción como ``aceite en
  las ruedas de la economía'' tiene serias deficiencias teóricas. 
\item Existen muchos costos económicos y no económicos
  no considerados --mala asignación del talento; inv. faraónicas
 con potencial de corrupción vs inv. necesarias con
 potencial social; elefantes blancos, etc. 
\item El costo del tiempo incurrido por las empresas en
  las diferentes etapas puede ser significativo --''construcción de la
  relación''. 
\item La evidencia empírica, finalmente, está fuertemente en contra de
  esta visión. Corrupción $\longrightarrow$ reduce IED, reduce GP
  Educ, + costo del K.  
\end{itemize}
\end{frame}


\begin{frame}\frametitle{Consecuencias de la corrupción}
\begin{itemize}\itemsep 10pt
\item A nivel teórico la corrupción tiene efectos
  negativos sobre el desarrollo por medio de varios canales. 
\begin{itemize} \itemsep 10pt
\item  Tres de estas vías son la asignación de de recursos en usos socialmente improductivos, los mayores costos de inversión debido al pago de sobornos y coimas, y el desfalco al Estado por parte de funcionarios públicos que disminuye el nivel disponible de ByS. 
\item Mauro (1995)  argumenta que la corrupción
  burocrática impacta negativamente en el crecimiento económico a
  través del efecto perjudicial sobre la inversión.
\end{itemize}
\item Más recientemente, se encuentra que países con alta corrupción
  tienen a tener menores niveles de GP en educación y salud. 
\end{itemize}
\end{frame}


\begin{frame}
\begin{figure}[htbp]
    \centering
    \includegraphics[scale=0.25]{wbcinv}
    \caption{Corrupción e inversión privada}
    \label{fig:corruptgastosocial}
  \end{figure}
\end{frame}





\begin{frame}
\begin{figure}[htbp]
    \centering
    \includegraphics[scale=0.5]{barometer2010}
    \caption{Usuarios que admiten haber pagado sobornos}
    \label{fig:corruptsobornos}
  \end{figure}
\end{frame}



\begin{frame}
\begin{figure}[htbp]
    \centering
    \includegraphics[scale=0.25]{wbcsalud}
    \caption{Corrupción y gasto social}
    \label{fig:corruptgastosocial}
  \end{figure}
\end{frame}

\begin{frame}\frametitle{Diferencias intra-país}
\begin{itemize}\itemsep 10pt
\item No sólo existen grandes diferencias en la corrupción entre los
  países; la corrupción varía mucho hacia adentro de los países. Un
  caso clásico de estudio es el de Italia y las regiones norte y sur. 
\item También en los países desarrollados existen variaciones en los
  niveles de corrupción entre las diferentes regiones; las
  explicaciones son a veces institucionales, culturales y
  etno-linguisticas. 
\item En los países de Africa es interesante ver como los niveles de
  corrupción y sobornos son generalmente mucho más altos alrededor de
  los límites. 
\end{itemize}

\end{frame}


\begin{frame}\frametitle{Diferencias intra-país (cont.)}
\begin{figure}[htbp]
    \centering
    \includegraphics[scale=0.6]{us-map-of-corruption}
    \caption{Diferencias regionales en la corrupción. Fuente: State
      Integrity Investigation}
    \label{fig:corruptregionsusa}
  \end{figure}
\end{frame}


\begin{frame}\frametitle{Diferencias intra-país (cont.)}
\begin{figure}[htbp]
    \centering
    \includegraphics[scale=0.32]{africaborders}
    \caption{Limites y etnias en Africa}
    \label{fig:africaborders}
  \end{figure}
\end{frame}



\begin{frame}\frametitle{¿Por qué los países tienen diferentes niveles
  de corrupción?}
\begin{itemize}\itemsep 10pt
\item Causas económicas y no-económicas
\item Entre las económicas, se encuentran niveles de ingreso,
  inflación, riqueza de recursos naturales. 
\item Entre las no-económicas, se tienen una gran cantidad de
  factores: libertad de prensa; larga exposición a condiciones
  democráticas; mayoría religiosa; instituciones
  coloniales; fragmentación etno-linguística; régimen de gobierno
\item En la gran mayoría de los estudios se coincide en que los
  principales determinantes cuantitativos son el nivel de ingreso,
  exposición democrática, libertad de prensa. 
\end{itemize}
\end{frame}


\begin{frame}\frametitle{Corrupción: causas (cont.)}
\begin{figure}[htbp]
    \centering
    \includegraphics[scale=0.3]{causascorrup}
 \end{figure}
\end{frame}


\begin{frame}\frametitle{Corrupción y libertad de prensa}
A pesar de ser una garantía constitucional en los estados democráticos, los niveles de libertad de prensa difieren mucho entre los países. \\ \medskip 
Existen restricciones de todo tipo que pueden agruparse en tres: \medskip 
\begin{itemize}\itemsep 15pt
\item restricciones legales y regulatorias
\item restricciones de tipo político
\item restricciones de tipo económico
\end{itemize}
\end{frame}


\begin{frame} \frametitle{Corrupción y libertad de prensa (cont.)}
\begin{itemize} \itemsep 10pt
\item En Italia, el primer ministro Berlusconi posee influencia sobre el 90\% de las transmisiones de TV
\item En Jordania y Malasia, la Constitución y otros códigos legales contienen disposiciones
extremas que restringen la libertad de los medios
\item Los niveles de violencia hacia los periodistas son particularmente altos en Colombia e Indonesia.
\item En los países de la escición de la USSR, muchos periodistas y cadenas locales han cedido a presiones de grupos de negocios
\end{itemize}
\end{frame}


\begin{frame} \frametitle{Corrupción y libertad de prensa (cont.)}
\begin{figure}[htbp]
    \centering
    \includegraphics[scale=0.3]{grafico6}
 \end{figure}
\end{frame}





\begin{frame}\frametitle{Corrupción y libertad de prensa (cont.)}
La idea central es que una prensa libre e independiente constituye un control sobre los excesos gubernamentales [Graber (1986), Pharr and Putnam (1997)]. \\ \medskip 
Este objetivo se logra por medio de dos tipos de actividades: \medskip 
\begin{itemize} \itemsep 15pt
\item reporte y difusión de noticias de actualidad
\item periodismo de investigación
\end{itemize}\medskip 
Pero la efectividad y eficacia de los medios se ve afectado por varios factores (corrupción en los medios; medios con influencias políticas; leyes y regulaciones restrictivas; presiones económicas)
\end{frame}



\begin{frame}\frametitle{Corrupción y libertad de prensa (cont.)}
En el trabajo se usan los siguientes datos: \medskip 
\begin{itemize}\itemsep 10pt
\item Datos agregados y desagregados sobre libertad de prensa de Freedom House. El ranking de FH va de 0 (libertad de prensa máxima) a 100 (libertad de prensa cero)
(http://www.freedomhouse.org)
\item Datos subjetivos de percepción de corrupción: Transparency International (TI) y International Countr Risk Guide (ICRG). El ranking de TI va de 0 (corrupción extrema)
a 10 (corrupción cero) (http://www.transparency.org/)
\item Gran cantidad de variables culturales, institucionales, históricas y políticas
\end{itemize}
\end{frame}


\begin{frame}\frametitle{Corrupción y libertad de prensa (cont.)}
\begin{figure}[htbp]
    \centering
    \includegraphics[scale=0.3]{tabla4}
 \end{figure}
\end{frame}



\begin{frame}\frametitle{Corrupción y libertad de prensa (cont.)}
\begin{figure}[htbp]
    \centering
    \includegraphics[scale=0.3]{tabla5}
 \end{figure}
\end{frame}



\begin{frame}\frametitle{Corrupción y libertad de prensa (cont.)}
Principales conclusiones: \medskip 
\begin{itemize} \itemsep 10pt
\item La libertad de prensa está asociada negativamente con la incidencia de corrupción (i.e. mayor libertad de prensa, menor corrupción)
\item Importa el tipo de restricciones a la prensa: mientras las restricciones de tipo político y económico son significativas y cuantitativamente importantes, las restricciones de tipo legal no lo son
\item Existe alguna indicación de que la causalidad va de una mayor libertad de prensa hacia menores niveles de corrupción –igualmente, no se puede descartar bidireccionalidad
\end{itemize}
\end{frame}




\begin{frame}\frametitle{Corrupción y libertad de prensa (cont.)}
Montesinos, asesor presidencial, guardaba un registro de todas las instancias de transacción corruptas, i.e. ``los vladivideos'' \medskip 
\begin{itemize}\itemsep 10pt
\item Registros detallados de pagos (sobornos) a jueces, políticos y a los medios $\longrightarrow$ ¿el
costo de comprar la democracia?
\item Evidencia de mayores pagos a los diferentes tipos de medios que a otros actores
institucionales
\item Un caso que muestra que los medios pueden resultar el control más eficiente y eficaz ante abusos del sector público
\end{itemize}
\end{frame}


\begin{frame}\frametitle{Diferencias en la corrupción (cont.)}
\begin{figure}[htbp]
  \centering
  \includegraphics[scale=0.5]{montesinos}
  \caption{La prueba}
  \label{fig:kk}
\end{figure}
\end{frame}



\begin{frame}\frametitle{Corrupción y libertad de prensa (cont.)}
\begin{figure}[htbp]
    \centering
    \includegraphics[scale=0.3]{grafico7}
 \end{figure}
\end{frame}


\begin{frame}\frametitle{Corrupción y libertad de prensa (cont.)}
\begin{figure}[htbp]
    \centering
    \includegraphics[scale=0.3]{grafico7bis}
 \end{figure}
\end{frame}


\begin{frame}\frametitle{Corrupción y libertad de prensa (cont.)}
\begin{figure}[htbp]
    \centering
    \includegraphics[scale=0.3]{grafico8}
 \end{figure}
\end{frame}



\begin{frame}\frametitle{Corrupción y libertad de prensa (cont.)}
\begin{figure}[htbp]
    \centering
    \includegraphics[scale=0.3]{grafico9}
 \end{figure}
\end{frame}



\begin{frame}\frametitle{Corrupción y libertad de prensa (cont.)}
\begin{figure}[htbp]
    \centering
    \includegraphics[scale=0.3]{grafico9bis}
 \end{figure}
\end{frame}







\end{document}



