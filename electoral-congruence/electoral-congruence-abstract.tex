% Options for packages loaded elsewhere
\PassOptionsToPackage{unicode}{hyperref}
\PassOptionsToPackage{hyphens}{url}
\PassOptionsToPackage{dvipsnames,svgnames*,x11names*}{xcolor}
%
\documentclass[
  12pt,
]{article}
\usepackage{amsmath,amssymb}
\usepackage{lmodern}
\usepackage{iftex}
\ifPDFTeX
  \usepackage[T1]{fontenc}
  \usepackage[utf8]{inputenc}
  \usepackage{textcomp} % provide euro and other symbols
\else % if luatex or xetex
  \usepackage{unicode-math}
  \defaultfontfeatures{Scale=MatchLowercase}
  \defaultfontfeatures[\rmfamily]{Ligatures=TeX,Scale=1}
\fi
% Use upquote if available, for straight quotes in verbatim environments
\IfFileExists{upquote.sty}{\usepackage{upquote}}{}
\IfFileExists{microtype.sty}{% use microtype if available
  \usepackage[]{microtype}
  \UseMicrotypeSet[protrusion]{basicmath} % disable protrusion for tt fonts
}{}
\makeatletter
\@ifundefined{KOMAClassName}{% if non-KOMA class
  \IfFileExists{parskip.sty}{%
    \usepackage{parskip}
  }{% else
    \setlength{\parindent}{0pt}
    \setlength{\parskip}{6pt plus 2pt minus 1pt}}
}{% if KOMA class
  \KOMAoptions{parskip=half}}
\makeatother
\usepackage{xcolor}
\IfFileExists{xurl.sty}{\usepackage{xurl}}{} % add URL line breaks if available
\IfFileExists{bookmark.sty}{\usepackage{bookmark}}{\usepackage{hyperref}}
\hypersetup{
  pdftitle={Electoral institutions, vote congruence and coattail effects in executive elections: The case of Argentina in 2003-2019},
  pdfauthor={Sebastián Freille},
  colorlinks=true,
  linkcolor={blue},
  filecolor={Maroon},
  citecolor={blue},
  urlcolor={blue},
  pdfcreator={LaTeX via pandoc}}
\urlstyle{same} % disable monospaced font for URLs
\usepackage[margin=1in]{geometry}
\usepackage{graphicx}
\makeatletter
\def\maxwidth{\ifdim\Gin@nat@width>\linewidth\linewidth\else\Gin@nat@width\fi}
\def\maxheight{\ifdim\Gin@nat@height>\textheight\textheight\else\Gin@nat@height\fi}
\makeatother
% Scale images if necessary, so that they will not overflow the page
% margins by default, and it is still possible to overwrite the defaults
% using explicit options in \includegraphics[width, height, ...]{}
\setkeys{Gin}{width=\maxwidth,height=\maxheight,keepaspectratio}
% Set default figure placement to htbp
\makeatletter
\def\fps@figure{htbp}
\makeatother
\setlength{\emergencystretch}{3em} % prevent overfull lines
\providecommand{\tightlist}{%
  \setlength{\itemsep}{0pt}\setlength{\parskip}{0pt}}
\setcounter{secnumdepth}{-\maxdimen} % remove section numbering
\ifLuaTeX
  \usepackage{selnolig}  % disable illegal ligatures
\fi

\title{Electoral institutions, vote congruence and coattail effects in
executive elections: The case of Argentina in 2003-2019}
\author{Sebastián Freille\footnote{Instituto de Economía y Finanzas,
  Facultad de Ciencias Económicas (FCE)-UNC. Email:
  \href{mailto:sfreille@unc.edu.ar}{\nolinkurl{sfreille@unc.edu.ar}}.
  Web: \url{https://sfreille.github.io}}}
\date{28 August 2022}

\begin{document}
\maketitle

\hypertarget{abstract}{%
\section{Abstract}\label{abstract}}

In this paper we study the relationship between electoral outcomes for
both federal and state level executive elections. Electoral outcomes for
different offices in multi-tiered systems are likely to be mutually
influenced through multiple channels. We explore two of this channels in
this paper: institutional design and coattail effects. Traditionally,
coattail effects have been studied between executive and legislative
elections for the same government level. Instead, we explore both
coattail effects and vote congruence for different-level office. Using
data disaggregated at the department-level comprising 5 (five) elections
during 2003-2019, we examine the relationship between votes for the
elected President and governor in every district. We find evidence of
vote dissimilarity between elected Presidents and governors and this is
particularly strong when the national executive election is contested.
Elected Presidents tend to increase their district-level relative
electoral strength vis-a-vis elected gobernors regardless of party and
coalition. This is consistent with anecdotal evidence and insights on
the characteristis of coalition-building between national and
sub-national governments.

\hypertarget{resumen}{%
\section{Resumen}\label{resumen}}

En este artículo estudiamos la relación entre los resultados electorales
para cargos ejecutivos nacionales y sub-nacionales. Los resultados
electorales para diferentes cargos en sistemas multinivel son
influenciados mutuamente a través de múltiples canales. Exploramos dos
de estos canales: el diseño institucional y los efectos de arrastre.
Tradicionalmente, los efectos de arrastre han sido estudiados entre
elecciones a diferentes cargos (ejecutivas y legislativas) para un mismo
nivel de gobierno. En este trabajo, exploramos los efectos de arrastre y
la congruencia del voto para iguales cargos de diferente nivel de
gobierno. Usando datos desagregados al nivel departamental para 5
(cinco) elecciones ejecutivas entre 2003 y 2019, examinamos la relación
entre los votos obtenidos por el Presidente y gobernador electo en cada
distrito. Encontramos evidencia de dismilaridad de votos entre los
Presidentes y gobernadores electos y esta es particularmente importante
cuando la elección nacional es disputada. Los Presidentes electos tienen
a mejorar su desempeño electoral a nivel de distrito relativo al de los
gobernadores electos independientemente de su partido y coalición. Esta
evidencia se corresponde con evidencia anecdótica y caracterizaciones
del proceso de construcción de coaliciones entre los gobiernos
nacionales y sub-nacionales en Argentina.

\end{document}
