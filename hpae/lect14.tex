% Options for packages loaded elsewhere
\PassOptionsToPackage{unicode}{hyperref}
\PassOptionsToPackage{hyphens}{url}
%
\documentclass[
  ignorenonframetext,
]{beamer}
\usepackage{pgfpages}
\setbeamertemplate{caption}[numbered]
\setbeamertemplate{caption label separator}{: }
\setbeamercolor{caption name}{fg=normal text.fg}
\beamertemplatenavigationsymbolsempty
% Prevent slide breaks in the middle of a paragraph
\widowpenalties 1 10000
\raggedbottom
\setbeamertemplate{part page}{
  \centering
  \begin{beamercolorbox}[sep=16pt,center]{part title}
    \usebeamerfont{part title}\insertpart\par
  \end{beamercolorbox}
}
\setbeamertemplate{section page}{
  \centering
  \begin{beamercolorbox}[sep=12pt,center]{part title}
    \usebeamerfont{section title}\insertsection\par
  \end{beamercolorbox}
}
\setbeamertemplate{subsection page}{
  \centering
  \begin{beamercolorbox}[sep=8pt,center]{part title}
    \usebeamerfont{subsection title}\insertsubsection\par
  \end{beamercolorbox}
}
\AtBeginPart{
  \frame{\partpage}
}
\AtBeginSection{
  \ifbibliography
  \else
    \frame{\sectionpage}
  \fi
}
\AtBeginSubsection{
  \frame{\subsectionpage}
}

\usepackage{amsmath,amssymb}
\usepackage{lmodern}
\usepackage{iftex}
\ifPDFTeX
  \usepackage[T1]{fontenc}
  \usepackage[utf8]{inputenc}
  \usepackage{textcomp} % provide euro and other symbols
\else % if luatex or xetex
  \usepackage{unicode-math}
  \defaultfontfeatures{Scale=MatchLowercase}
  \defaultfontfeatures[\rmfamily]{Ligatures=TeX,Scale=1}
\fi
% Use upquote if available, for straight quotes in verbatim environments
\IfFileExists{upquote.sty}{\usepackage{upquote}}{}
\IfFileExists{microtype.sty}{% use microtype if available
  \usepackage[]{microtype}
  \UseMicrotypeSet[protrusion]{basicmath} % disable protrusion for tt fonts
}{}
\makeatletter
\@ifundefined{KOMAClassName}{% if non-KOMA class
  \IfFileExists{parskip.sty}{%
    \usepackage{parskip}
  }{% else
    \setlength{\parindent}{0pt}
    \setlength{\parskip}{6pt plus 2pt minus 1pt}}
}{% if KOMA class
  \KOMAoptions{parskip=half}}
\makeatother
\usepackage{xcolor}
\newif\ifbibliography
\setlength{\emergencystretch}{3em} % prevent overfull lines
\setcounter{secnumdepth}{-\maxdimen} % remove section numbering


\providecommand{\tightlist}{%
  \setlength{\itemsep}{0pt}\setlength{\parskip}{0pt}}\usepackage{longtable,booktabs,array}
\usepackage{calc} % for calculating minipage widths
\usepackage{caption}
% Make caption package work with longtable
\makeatletter
\def\fnum@table{\tablename~\thetable}
\makeatother
\usepackage{graphicx}
\makeatletter
\def\maxwidth{\ifdim\Gin@nat@width>\linewidth\linewidth\else\Gin@nat@width\fi}
\def\maxheight{\ifdim\Gin@nat@height>\textheight\textheight\else\Gin@nat@height\fi}
\makeatother
% Scale images if necessary, so that they will not overflow the page
% margins by default, and it is still possible to overwrite the defaults
% using explicit options in \includegraphics[width, height, ...]{}
\setkeys{Gin}{width=\maxwidth,height=\maxheight,keepaspectratio}
% Set default figure placement to htbp
\makeatletter
\def\fps@figure{htbp}
\makeatother

\makeatletter
\makeatother
\makeatletter
\makeatother
\makeatletter
\@ifpackageloaded{caption}{}{\usepackage{caption}}
\AtBeginDocument{%
\ifdefined\contentsname
  \renewcommand*\contentsname{Table of contents}
\else
  \newcommand\contentsname{Table of contents}
\fi
\ifdefined\listfigurename
  \renewcommand*\listfigurename{List of Figures}
\else
  \newcommand\listfigurename{List of Figures}
\fi
\ifdefined\listtablename
  \renewcommand*\listtablename{List of Tables}
\else
  \newcommand\listtablename{List of Tables}
\fi
\ifdefined\figurename
  \renewcommand*\figurename{Figure}
\else
  \newcommand\figurename{Figure}
\fi
\ifdefined\tablename
  \renewcommand*\tablename{Table}
\else
  \newcommand\tablename{Table}
\fi
}
\@ifpackageloaded{float}{}{\usepackage{float}}
\floatstyle{ruled}
\@ifundefined{c@chapter}{\newfloat{codelisting}{h}{lop}}{\newfloat{codelisting}{h}{lop}[chapter]}
\floatname{codelisting}{Listing}
\newcommand*\listoflistings{\listof{codelisting}{List of Listings}}
\makeatother
\makeatletter
\@ifpackageloaded{caption}{}{\usepackage{caption}}
\@ifpackageloaded{subcaption}{}{\usepackage{subcaption}}
\makeatother
\makeatletter
\@ifpackageloaded{tcolorbox}{}{\usepackage[many]{tcolorbox}}
\makeatother
\makeatletter
\@ifundefined{shadecolor}{\definecolor{shadecolor}{rgb}{.97, .97, .97}}
\makeatother
\makeatletter
\makeatother
\ifLuaTeX
  \usepackage{selnolig}  % disable illegal ligatures
\fi
\IfFileExists{bookmark.sty}{\usepackage{bookmark}}{\usepackage{hyperref}}
\IfFileExists{xurl.sty}{\usepackage{xurl}}{} % add URL line breaks if available
\urlstyle{same} % disable monospaced font for URLs
\hypersetup{
  pdftitle={U8. El nacimiento de la macroeconomía},
  hidelinks,
  pdfcreator={LaTeX via pandoc}}

\title{U8. El nacimiento de la macroeconomía}
\subtitle{Historia del Pensamiento y del Análisis Económico}
\author{\textbf{Sebastián Freille}\\
sfreille@unc.edu.ar\\
Licenciatura en Economía\\
FCE-UNC}
\date{}

\begin{document}
\frame{\titlepage}
\ifdefined\Shaded\renewenvironment{Shaded}{\begin{tcolorbox}[sharp corners, interior hidden, borderline west={3pt}{0pt}{shadecolor}, enhanced, breakable, frame hidden, boxrule=0pt]}{\end{tcolorbox}}\fi

\hypertarget{el-siglo-xx-nacimiento-desarrollo-y-evoluciuxf3n-de-la-macroeconomuxeda}{%
\section{El siglo XX: nacimiento, desarrollo y evolución de la
macroeconomía}\label{el-siglo-xx-nacimiento-desarrollo-y-evoluciuxf3n-de-la-macroeconomuxeda}}

\begin{frame}{Contexto}
\protect\hypertarget{contexto}{}
\begin{itemize}
\tightlist
\item
  El nacimiento de la macroeconomía como campo específico de la economía
  se da en la década de 1930s con la publicación de dos obras
  fundamentales del que es por muchos considerado la figura económica
  más importante del siglo XX, John Maynard Keynes

  \begin{itemize}
  \tightlist
  \item
    \emph{``Tratado sobre el Dinero''} (1930)
  \item
    \emph{``Teoría General de la Ocupación, el Interés y el Dinero''}
    (1936)
  \end{itemize}
\item
  Ambos libros revelan el principal interés y objetivo de Keynes
  \(\longrightarrow\) explicar las fluctuaciones en el producto y empleo
  que caracterizan al ciclo económico.
\end{itemize}
\end{frame}

\begin{frame}{Contexto (cont.)}
\protect\hypertarget{contexto-cont.}{}
\begin{quote}
He llamado a este libro \emph{Teoría general de la ocupación, el interés
y el dinero}, recalcando el sufijo \emph{general}, con objeto de que el
título sirva para contrastar mis argumentos y conclusiones con los de la
teoría \emph{clásica}\ldots{} \ldots Sostendré que los postulados de la
teoría clásica sólo son aplicables a un caso especial, y no en general,
porque las condiciones que supone son un caso extremo de todas las
posiciones posibles de equilibrio. Mas aún, las características del caso
especial supuesto por la teoría clásica no son las de la sociedad
económica en que hoy vivimos, razón por la que sus enseñanzas engañan y
son desastrosas si intentamos aplicarlas a los hechos reales. {[}Keynes,
J.M. (1936), \textbf{Teoría general de la ocupación, el interés y el
dinero}, pp.~15{]}
\end{quote}
\end{frame}

\begin{frame}{Contexto (cont.)}
\protect\hypertarget{contexto-cont.-1}{}
\begin{itemize}
\tightlist
\item
  Varias razones confluyen para surgimiento de interes por macroeconomía

  \begin{itemize}
  \tightlist
  \item
    Marcha de acontecimientos económicos --ciclo de crédito 1920s,
    crisis 1929
  \item
    Revolución de la estadística liderada por Clark y Kuznets
  \item
    Experiencia, intereses y situaciones personales (Hayek y los efectos
    de la inflación sobre Viena)
  \end{itemize}
\item
  ¿Existían ideas y análisis macroeconómico antes de Keynes?

  \begin{itemize}
  \tightlist
  \item
    Tradiciones de pensamiento que se ocuparon de cuestiones que hoy
    consideramos parte de la macroeconomía

    \begin{itemize}
    \tightlist
    \item
      ciclos de negocios (Jevons, Juglar, Mitchell)
    \item
      teoría monetaria (Hume, Thornton, Ricardo, Wicksell y Fisher)
    \end{itemize}
  \end{itemize}
\end{itemize}
\end{frame}

\begin{frame}{Contexto (cont.)}
\protect\hypertarget{contexto-cont.-2}{}
\begin{itemize}
\tightlist
\item
  Cierta idea de que la macroeconomía se desarrolla por una serie de
  batallas, revoluciones y contrarrevoluciones

  \begin{itemize}
  \tightlist
  \item
    Blanchard (2000) argumenta que en realidad ha sido un proceso de
    acumulación de conocimiento estable
  \end{itemize}
\item
  Propone la siguiente periodización

  \begin{enumerate}
  \tightlist
  \item
    Pre-1940: período de exploración
  \item
    De 1940 a 1980: período de consolidacio´n
  \item
    Post-1980: nuevo período de exploración
  \end{enumerate}
\end{itemize}
\end{frame}

\begin{frame}{Períodos}
\protect\hypertarget{peruxedodos}{}
\begin{itemize}
\tightlist
\item
  Enfoque casi exclusivamente de corto plazo. Dominado por estudio de 2
  (dos) campos paralelos: 1) teoría monetaria; 2) teoría del ciclo
  económico.
\item
  Durante mucho tiempo estos dos campos aislados con distinto grado de
  desarrollo
\item
  Contribuciones metodológicas de 1920s/1930s (Keynes) permiten avanzar
  en la integración de ambos campos
\end{itemize}
\end{frame}

\begin{frame}{Períodos (cont.)}
\protect\hypertarget{peruxedodos-cont.}{}
\begin{itemize}
\tightlist
\item
  Se desarrolla conceptual, analítica y esquemáticamente el modelo
  keynesiano a partir de la formulación IS-LM (Hicks-Hansen)
\item
  Paradigma útil como estándar unificado de la teoría macroeconómica y
  lo suficientemente flexible como para incorporar extensiones y
  ampliaciones (curva de Phillips, ER, relaciones comportamentales)
\item
  Este período culmina con el desarrollo e incorporación de fundamentos
  microeconómicos
\end{itemize}
\end{frame}

\begin{frame}{Períodos (cont.)}
\protect\hypertarget{peruxedodos-cont.-1}{}
\begin{itemize}
\tightlist
\item
  Nuevo período de exploración que cuestiona radicalmente algunas de las
  cuasi-verdades de la ortodoxia reinante
\item
  Las dos principales rupturas con esa ortodoxia macroeconómica fueron

  \begin{itemize}
  \tightlist
  \item
    idea de que los ciclos económicos provocados por perturbaciones de
    demanda son siempre indeseables y de que la economía regresaría a
    una situación de tasa natural
  \item
    idea de que las interacciones económicas de los agentes económicos
    ocurren principalmente en entornos de competencia e información
    perfecta en los mercados de bienes, trabajo y crédito
  \end{itemize}
\end{itemize}
\end{frame}

\begin{frame}{Período I: Exploración}
\protect\hypertarget{peruxedodo-i-exploraciuxf3n}{}
\begin{itemize}
\tightlist
\item
  Hasta la publicación de la \emph{Teoría General}, el cuerpo teórico y
  analítico de la economía y consolidado la economía clásica y
  neoclásica {[}Keynes consideraba a los neoclásicos continuadores y
  seguidores de los clásicos{]}

  \begin{itemize}
  \tightlist
  \item
    cuerpo relativamente homogéneo en términos de orientación y mensaje
    principal

    \begin{itemize}
    \tightlist
    \item
      confianza en mecanismos espontáneos de ajuste del mercado como
      medio de mantener equilibrio de pleno empleo
    \end{itemize}
  \end{itemize}
\item
  Keynes primero ``construye'' su enemigo, le da forma y luego pelea.
\end{itemize}
\end{frame}

\begin{frame}{Período I: Exploración (cont.)}
\protect\hypertarget{peruxedodo-i-exploraciuxf3n-cont.}{}
\begin{itemize}
\tightlist
\item
  A principios del siglo XX no había una teoría única y formalizada
  sobre la determinación del nivel de empleo y producto agregados

  \begin{itemize}
  \tightlist
  \item
    si una variedad de hipótesis y explicaciones sobre la naturaleza y
    características de los ciclos económicos
  \end{itemize}
\item
  Aún así, era mucho más rudimentaria y menos cohesionada que el estadío
  que había alcanzado la teoría microeconómica
\item
  De hecho, gran parte de las ideas sobre macroeconomía de la escuela
  clásica son sistematizadas a partir de la obra de Keynes
\end{itemize}
\end{frame}

\begin{frame}{Período I: la escuela clásica}
\protect\hypertarget{peruxedodo-i-la-escuela-cluxe1sica}{}
\begin{itemize}
\tightlist
\item
  Los clásicos eran plenamente conscientes de que una economía
  capitalista podía desviarse de su nivel de equilibrio de producto y
  empleo
\item
  Pero en general suponían que estas desviaciones serían temporales y de
  corta duración
\item
  En esta situación, no había ningún rol para la injerencia estatal en
  relación a cuestiones de estabilización del producto y el empleo

  \begin{itemize}
  \tightlist
  \item
    consideraban que el equilibrio de PE era el estado normal de los
    mercados (norma) y que las fluctuaciones eran la excepción
  \end{itemize}
\end{itemize}
\end{frame}

\begin{frame}{Período I: la escuela clásica (cont.)}
\protect\hypertarget{peruxedodo-i-la-escuela-cluxe1sica-cont.}{}
\begin{itemize}
\tightlist
\item
  La macroeconomía de la escuela clásica puede resumirse en la
  interacción de 3 (tres) elementos centrales:

  \begin{enumerate}
  \tightlist
  \item
    Teoría clásica de la determinación del empleo y del producto
  \item
    La ley de Say de los mercados
  \item
    La teoría cuantitativa del dinero
  \end{enumerate}
\item
  La interacción de los primeros resulta en que las variables reales
  --producto real, \(Y\); nivel de empleo, \(L\); salario real, \(W/P\);
  y tasa de interés real, \(r\)- se determinan en los mercados de bienes
  y trabajo

  \begin{itemize}
  \tightlist
  \item
    \(Y\) es una función de \(L\), \(K\) y \(A\) (productividad) y hay
    una relación directa entre \(L\) y \(Y\)
  \end{itemize}
\end{itemize}
\end{frame}

\begin{frame}{Período I: la escuela clásica (cont.)}
\protect\hypertarget{peruxedodo-i-la-escuela-cluxe1sica-cont.-1}{}
\begin{itemize}
\tightlist
\item
  Nivel de empleo se determina a partir de igual el costo marginal de
  emplear unidades adicionales de \(L\) y del ingreso marginal. Es
  decir, donde \(PML_{i}P_{i}=W_{i}\) lo que es igual a
  \(PML_{i}=\frac{W_{i}}{P_{i}}\)
\item
  El salario real se ajusta ante excesos de oferta (baja) y demanda de
  trabajo (sube)

  \begin{itemize}
  \tightlist
  \item
    Keynes dijo --los postulados clásicos no admiten la posibilidad de
    ``desempleo involuntario''
  \end{itemize}
\item
  Si salarios reales por encima del equilibrio \(\longrightarrow\) para
  los clásicos debía bajarse el salario nominal
\end{itemize}
\end{frame}

\begin{frame}{Período I: la escuela clásica (cont.)}
\protect\hypertarget{peruxedodo-i-la-escuela-cluxe1sica-cont.-2}{}
\begin{itemize}
\tightlist
\item
  Tasa de interés natural equilibra ahorro e inversión reales.
  Suponiendo 2 (dos) sectores --hogares y empresas-, tenemos que:
\end{itemize}

\textbackslash begin\{align\} E=C(r)+I(r)\&=Y \textbackslash{}
Y-C(r)\&=S(r) \textbackslash end\{equation\}

\begin{itemize}
\tightlist
\item
  todas las variables son función de la tasa de interés real
\end{itemize}
\end{frame}

\begin{frame}{Período I: la escuela clásica (cont.)}
\protect\hypertarget{peruxedodo-i-la-escuela-cluxe1sica-cont.-3}{}
\begin{itemize}
\tightlist
\item
  El modelo se cerraba con la ecuación cuantitativa del dinero que en
  formulación de Marshall era
\end{itemize}

\begin{equation}
M=kPY
\end{equation}

\begin{itemize}
\tightlist
\item
  donde \(k\) es la inversa de la velocidad de circulacion
\end{itemize}
\end{frame}

\begin{frame}{Período I: la escuela clásica (cont.)}
\protect\hypertarget{peruxedodo-i-la-escuela-cluxe1sica-cont.-4}{}
\begin{itemize}
\tightlist
\item
  El ``modelo macroeconómico clásico'' implicaba un resultado muy
  importante \(\longrightarro\) \textbf{dicotomía clásica entre sector
  real y monetario}

  \begin{itemize}
  \tightlist
  \item
    cambios en las variables nominales no afectarán los valores de
    equilibrio de las variables reales
  \end{itemize}
\item
  Y de aquí se deriva el famoso postulado sobre la neutralidad del
  dinero en la economía clásica
\end{itemize}
\end{frame}

\begin{frame}{Período I: la macroeconomía keynesiana}
\protect\hypertarget{peruxedodo-i-la-macroeconomuxeda-keynesiana}{}
\begin{itemize}
\tightlist
\item
  Para Keynes, \(Y\) depende del volumen de \(L\) pero es posible la
  existencia de desempleo involuntario en el equilibrio
\item
  Una diferencia clave entre clasicos y keynesianos era el mecanismo de
  ajuste \(\longrightarrow\) para los clásicos el ajuste se daba por
  precios, para keynesianos por cantidades
\item
  Principal elemento innovador en la teoría keynesiana
  \(\longrightarrow\) introducción del concepto de \textbf{demanda
  efectiva}
\end{itemize}
\end{frame}

\begin{frame}{Período I: la macroeconomía keynesiana (cont.)}
\protect\hypertarget{peruxedodo-i-la-macroeconomuxeda-keynesiana-cont.}{}
\begin{quote}
El principio de la demanda efectiva estable que en una economía cerrada
con capacidad ociosa, el nivel de ingreso, \(Y\) (y por tanto el nivel
de empleo) viene determinado por el gasto agregado planeado, \(E\), que
consiste de dos componenetes, el gasto de consumo de los hogares, \(C\)
y el gasto de inversión de las empresas, \(I\).
\end{quote}
\end{frame}

\begin{frame}{Período I: la macroeconomía keynesiana (cont.)}
\protect\hypertarget{peruxedodo-i-la-macroeconomuxeda-keynesiana-cont.-1}{}
\begin{itemize}
\tightlist
\item
  Por tanto:
\end{itemize}

\begin{equation}
E=C+I 
\end{equation}

\begin{itemize}
\tightlist
\item
  Note la gran diferencia \(\longrightarrow\) en modelo keynesiano \(C\)
  es endógeno (depende de \(Y\)), y la inversión depende de la
  rentabilidad esperada de la inversión y la tasa de interés que
  representa el costo de pedir prestado fondos --i.e eficiencia marginal
  del capital

  \begin{itemize}
  \tightlist
  \item
    Keynes era bastante escéptico sobre el rol que la tasa de interés
    tenía sobre \(I\) --incertidumbre, expectativas
  \end{itemize}
\end{itemize}
\end{frame}

\begin{frame}{Período I: la macroeconomía keynesiana (cont.)}
\protect\hypertarget{peruxedodo-i-la-macroeconomuxeda-keynesiana-cont.-2}{}
\begin{itemize}
\tightlist
\item
\end{itemize}
\end{frame}



\end{document}
