% Options for packages loaded elsewhere
\PassOptionsToPackage{unicode}{hyperref}
\PassOptionsToPackage{hyphens}{url}
%
\documentclass[
]{article}
\usepackage{amsmath,amssymb}
\usepackage{lmodern}
\usepackage{iftex}
\ifPDFTeX
  \usepackage[T1]{fontenc}
  \usepackage[utf8]{inputenc}
  \usepackage{textcomp} % provide euro and other symbols
\else % if luatex or xetex
  \usepackage{unicode-math}
  \defaultfontfeatures{Scale=MatchLowercase}
  \defaultfontfeatures[\rmfamily]{Ligatures=TeX,Scale=1}
\fi
% Use upquote if available, for straight quotes in verbatim environments
\IfFileExists{upquote.sty}{\usepackage{upquote}}{}
\IfFileExists{microtype.sty}{% use microtype if available
  \usepackage[]{microtype}
  \UseMicrotypeSet[protrusion]{basicmath} % disable protrusion for tt fonts
}{}
\makeatletter
\@ifundefined{KOMAClassName}{% if non-KOMA class
  \IfFileExists{parskip.sty}{%
    \usepackage{parskip}
  }{% else
    \setlength{\parindent}{0pt}
    \setlength{\parskip}{6pt plus 2pt minus 1pt}}
}{% if KOMA class
  \KOMAoptions{parskip=half}}
\makeatother
\usepackage{xcolor}
\IfFileExists{xurl.sty}{\usepackage{xurl}}{} % add URL line breaks if available
\IfFileExists{bookmark.sty}{\usepackage{bookmark}}{\usepackage{hyperref}}
\hypersetup{
  pdftitle={Economía Política Internacional {[}UCC{]}},
  pdfauthor={Sebastián Freille},
  hidelinks,
  pdfcreator={LaTeX via pandoc}}
\urlstyle{same} % disable monospaced font for URLs
\usepackage[margin=1in]{geometry}
\usepackage{graphicx}
\makeatletter
\def\maxwidth{\ifdim\Gin@nat@width>\linewidth\linewidth\else\Gin@nat@width\fi}
\def\maxheight{\ifdim\Gin@nat@height>\textheight\textheight\else\Gin@nat@height\fi}
\makeatother
% Scale images if necessary, so that they will not overflow the page
% margins by default, and it is still possible to overwrite the defaults
% using explicit options in \includegraphics[width, height, ...]{}
\setkeys{Gin}{width=\maxwidth,height=\maxheight,keepaspectratio}
% Set default figure placement to htbp
\makeatletter
\def\fps@figure{htbp}
\makeatother
\setlength{\emergencystretch}{3em} % prevent overfull lines
\providecommand{\tightlist}{%
  \setlength{\itemsep}{0pt}\setlength{\parskip}{0pt}}
\setcounter{secnumdepth}{-\maxdimen} % remove section numbering
\ifLuaTeX
  \usepackage{selnolig}  % disable illegal ligatures
\fi

\title{Economía Política Internacional {[}UCC{]}}
\author{Sebastián Freille}
\date{(2021-09-02)}

\begin{document}
\maketitle

\hypertarget{bienvenida}{%
\subsection{Bienvenida}\label{bienvenida}}

Este es un \textbf{sitio de clases interactivas} de la materia Economía
Política Internacional (cátedra A) del cuarto año de las carreras de
Licenciatura en Ciencia Política y Licenciatura en Relaciones
Internacionales de la Facultad de Ciencia Política y Relaciones
Internacionales de la Universidad Católica de Córdoba.

\hypertarget{aspectos-organizativos}{%
\subsection{Aspectos organizativos}\label{aspectos-organizativos}}

En esta sección presentaremos los principales aspectos generales y
organizativos de la materia.

\hypertarget{equipo-de-cuxe1tedra}{%
\subsection{Equipo de cátedra}\label{equipo-de-cuxe1tedra}}

\begin{itemize}
\tightlist
\item
  \textbf{Profesor Titular: FREILLE, Sebastián}

  \begin{itemize}
  \tightlist
  \item
    Lic en Economía (UNC)
  \item
    PhD en Economía (University of Nottingham)
  \end{itemize}
\item
  \textbf{Profesora Adjunta: GORONDY NOVAK, Melisa}

  \begin{itemize}
  \tightlist
  \item
    Lic en Relaciones Internacionales (UCC)
  \item
    MA en Ciencias Sociales ``Global Studies Programme'' (Universidad
    Albert Ludwigs-FLACSO-Jawarhalal Nehru University)
  \end{itemize}
\end{itemize}

\hypertarget{clases-y-dictado}{%
\subsection{Clases y dictado}\label{clases-y-dictado}}

El curso se dicta íntegramente de manera virtual \textbf{usando una
combinación de metodologías}. Las principales serán:

\begin{enumerate}
\def\labelenumi{\arabic{enumi}.}
\tightlist
\item
  Clases virtuales sincrónicas
\item
  Material digitalizado y audiovisual en aula virtual
\item
  Encuentros sincrónicos semanales de consulta y revisión
\end{enumerate}

\hypertarget{clases-y-dictado-cont.}{%
\subsection{Clases y dictado (cont.)}\label{clases-y-dictado-cont.}}

Los horarios de los encuentros sincrónicos semanales serán:

\begin{enumerate}
\def\labelenumi{\arabic{enumi}.}
\tightlist
\item
  Martes de 16:00 a 18:00 {[}SF{]}
\item
  Miércoles de 17:00 a 19:00 {[}MGN{]}
\item
  Encuentros virtuales a confirmar
\end{enumerate}

\hypertarget{objetivo-general}{%
\subsection{Objetivo general}\label{objetivo-general}}

El objetivo de la materia es estudiar y analizar la interacción entre
las relaciones económicas internacionales de los países, las
instituciones políticas y el proceso de agregación de preferencias de
múltiples actores y las políticas económicas resultantes de esa
interacción.

\hypertarget{objetivos-especuxedficos}{%
\subsection{Objetivos específicos}\label{objetivos-especuxedficos}}

\begin{itemize}
\tightlist
\item
  Estimular y desarrollar en los estudiantes la capacidad analítica para
  el estudio de los principales fenómenos económicos internacionales.
\item
  Proponer un enfoque y abordaje positivo que permita un análisis
  riguroso de los temas
\item
  Contribuir a una mayor integración de los elementos económicos y
  políticos que interactúan en la configuración de las relaciones
  económicas internacionales
\item
  Promover en los estudiantes una visión reflexiva y crítica de los
  fenómenos estudiados
\end{itemize}

\hypertarget{programa-de-la-materia}{%
\subsection{Programa de la materia}\label{programa-de-la-materia}}

\begin{enumerate}
\def\labelenumi{\arabic{enumi}.}
\tightlist
\item
  Unidad 1: Introducción: Economía y política. Métodos y problemas
\item
  Unidad 2: Política comercial: Factores y sectores. Votantes y
  políticos. Instituciones
\item
  Unidad 3: Integración de mercados globales
\item
  Unidad 4: Economía política de la redistribución
\item
  Unidad 5: Tópicos globales de economía política
\end{enumerate}

\hypertarget{bibliografuxeda-y-textos}{%
\subsection{Bibliografía y textos}\label{bibliografuxeda-y-textos}}

\begin{itemize}
\tightlist
\item
  La cátedra no sigue un texto único
\item
  Se dará una bibliografía detallada de papers y extractos de libro
\item
  Hay bibliografía en castellano pero mayoritariamente es en inglés
\end{itemize}

\hypertarget{evaluaciuxf3n}{%
\subsection{Evaluación}\label{evaluaciuxf3n}}

\begin{itemize}
\tightlist
\item
  La evaluación tendrá dos instancias: parciales y finales. Habrá 2
  (dos) parciales y se podrá recuperar hasta uno ya sea por aplazo o
  ausencia.
\item
  El alumno obtendrá la \textbf{condición de regular (R) aprobando 2
  (dos) parciales}. En caso contrario, quedará libre parciales
  (LP)/libre asistencia (LA).
\item
  La evaluación final comprenderá todos los temas del programa y la
  modalidad a definir.
\item
  Cada evaluación (parcial/final) se aprobará con el 50\% del puntaje
  total
\end{itemize}

\hypertarget{fechas-importantes}{%
\subsection{Fechas importantes}\label{fechas-importantes}}

\begin{itemize}
\tightlist
\item
  Parcial \#1: 29-SEP-2021 (miércoles)
\item
  Parcial \#2: 02-NOV-2021 (martes)
\item
  Parcial Recuperatorio: 10-NOV-2021 (miércoles)
\end{itemize}

\hypertarget{puntos-de-encuentro}{%
\subsection{Puntos de encuentro}\label{puntos-de-encuentro}}

\begin{enumerate}
\def\labelenumi{\arabic{enumi}.}
\tightlist
\item
  Aula virtual de la materia:
  \href{https://campusvirtual.ucc.edu.ar/course/view.php?id=2037}{Aula
  virtual}
\item
  Web de S Freille: \href{https://sfreille.weebly.com/epin.html}{Página
  de Economía Política Internacional}
\item
  Salas Zoom

  \begin{itemize}
  \tightlist
  \item
    Freille:
    \href{https://us02web.zoom.us/j/86047575122?pwd=YmFQSnRCcnNqenVzMVV4Z1lMODFnZz09}{Sala
    de Zoom}
  \item
    Gorondy Novak: a comunicar
  \end{itemize}
\item
  Mail y otros

  \begin{itemize}
  \tightlist
  \item
    Se responden mails y mensajes en foros (no usamos grupos de
    whatsapp)
  \item
    Posible uso de Instagram Live (o similar) para sesiones de consulta
    cortas
  \end{itemize}
\end{enumerate}

\hypertarget{resumiendo}{%
\subsection{Resumiendo}\label{resumiendo}}

\begin{enumerate}
\def\labelenumi{\arabic{enumi}.}
\tightlist
\item
  Usen los espacios y metodologías dispuestos por la cátedra
\item
  Propuesta modelo de aula invertida (\textbf{flipped classroom})
  --estudio descentralizado (estudiantes) y guía/tutor (profesor)
\item
  \textbf{Dificultad media/alta} pero la filosofía de la catedra es:

  \begin{enumerate}
  \def\labelenumii{\arabic{enumii}.}
  \tightlist
  \item
    Explicamos los temas del programa
  \item
    Evaluamos sólo lo que explicamos
  \item
    Explicamos cómo evaluamos y cómo corregimos
  \item
    Mostramos evaluaciones y devoluciones
  \end{enumerate}
\end{enumerate}

\end{document}
