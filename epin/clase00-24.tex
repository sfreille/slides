% Options for packages loaded elsewhere
\PassOptionsToPackage{unicode}{hyperref}
\PassOptionsToPackage{hyphens}{url}
%
\documentclass[
  ignorenonframetext,
]{beamer}
\usepackage{pgfpages}
\setbeamertemplate{caption}[numbered]
\setbeamertemplate{caption label separator}{: }
\setbeamercolor{caption name}{fg=normal text.fg}
\beamertemplatenavigationsymbolsempty
% Prevent slide breaks in the middle of a paragraph
\widowpenalties 1 10000
\raggedbottom
\setbeamertemplate{part page}{
  \centering
  \begin{beamercolorbox}[sep=16pt,center]{part title}
    \usebeamerfont{part title}\insertpart\par
  \end{beamercolorbox}
}
\setbeamertemplate{section page}{
  \centering
  \begin{beamercolorbox}[sep=12pt,center]{part title}
    \usebeamerfont{section title}\insertsection\par
  \end{beamercolorbox}
}
\setbeamertemplate{subsection page}{
  \centering
  \begin{beamercolorbox}[sep=8pt,center]{part title}
    \usebeamerfont{subsection title}\insertsubsection\par
  \end{beamercolorbox}
}
\AtBeginPart{
  \frame{\partpage}
}
\AtBeginSection{
  \ifbibliography
  \else
    \frame{\sectionpage}
  \fi
}
\AtBeginSubsection{
  \frame{\subsectionpage}
}

\usepackage{amsmath,amssymb}
\usepackage{lmodern}
\usepackage{iftex}
\ifPDFTeX
  \usepackage[T1]{fontenc}
  \usepackage[utf8]{inputenc}
  \usepackage{textcomp} % provide euro and other symbols
\else % if luatex or xetex
  \usepackage{unicode-math}
  \defaultfontfeatures{Scale=MatchLowercase}
  \defaultfontfeatures[\rmfamily]{Ligatures=TeX,Scale=1}
\fi
% Use upquote if available, for straight quotes in verbatim environments
\IfFileExists{upquote.sty}{\usepackage{upquote}}{}
\IfFileExists{microtype.sty}{% use microtype if available
  \usepackage[]{microtype}
  \UseMicrotypeSet[protrusion]{basicmath} % disable protrusion for tt fonts
}{}
\makeatletter
\@ifundefined{KOMAClassName}{% if non-KOMA class
  \IfFileExists{parskip.sty}{%
    \usepackage{parskip}
  }{% else
    \setlength{\parindent}{0pt}
    \setlength{\parskip}{6pt plus 2pt minus 1pt}}
}{% if KOMA class
  \KOMAoptions{parskip=half}}
\makeatother
\usepackage{xcolor}
\newif\ifbibliography
\setlength{\emergencystretch}{3em} % prevent overfull lines
\setcounter{secnumdepth}{-\maxdimen} % remove section numbering


\providecommand{\tightlist}{%
  \setlength{\itemsep}{0pt}\setlength{\parskip}{0pt}}\usepackage{longtable,booktabs,array}
\usepackage{calc} % for calculating minipage widths
\usepackage{caption}
% Make caption package work with longtable
\makeatletter
\def\fnum@table{\tablename~\thetable}
\makeatother
\usepackage{graphicx}
\makeatletter
\def\maxwidth{\ifdim\Gin@nat@width>\linewidth\linewidth\else\Gin@nat@width\fi}
\def\maxheight{\ifdim\Gin@nat@height>\textheight\textheight\else\Gin@nat@height\fi}
\makeatother
% Scale images if necessary, so that they will not overflow the page
% margins by default, and it is still possible to overwrite the defaults
% using explicit options in \includegraphics[width, height, ...]{}
\setkeys{Gin}{width=\maxwidth,height=\maxheight,keepaspectratio}
% Set default figure placement to htbp
\makeatletter
\def\fps@figure{htbp}
\makeatother

\makeatletter
\makeatother
\makeatletter
\makeatother
\makeatletter
\@ifpackageloaded{caption}{}{\usepackage{caption}}
\AtBeginDocument{%
\ifdefined\contentsname
  \renewcommand*\contentsname{Table of contents}
\else
  \newcommand\contentsname{Table of contents}
\fi
\ifdefined\listfigurename
  \renewcommand*\listfigurename{List of Figures}
\else
  \newcommand\listfigurename{List of Figures}
\fi
\ifdefined\listtablename
  \renewcommand*\listtablename{List of Tables}
\else
  \newcommand\listtablename{List of Tables}
\fi
\ifdefined\figurename
  \renewcommand*\figurename{Figure}
\else
  \newcommand\figurename{Figure}
\fi
\ifdefined\tablename
  \renewcommand*\tablename{Table}
\else
  \newcommand\tablename{Table}
\fi
}
\@ifpackageloaded{float}{}{\usepackage{float}}
\floatstyle{ruled}
\@ifundefined{c@chapter}{\newfloat{codelisting}{h}{lop}}{\newfloat{codelisting}{h}{lop}[chapter]}
\floatname{codelisting}{Listing}
\newcommand*\listoflistings{\listof{codelisting}{List of Listings}}
\makeatother
\makeatletter
\@ifpackageloaded{caption}{}{\usepackage{caption}}
\@ifpackageloaded{subcaption}{}{\usepackage{subcaption}}
\makeatother
\makeatletter
\@ifpackageloaded{tcolorbox}{}{\usepackage[many]{tcolorbox}}
\makeatother
\makeatletter
\@ifundefined{shadecolor}{\definecolor{shadecolor}{rgb}{.97, .97, .97}}
\makeatother
\makeatletter
\makeatother
\ifLuaTeX
  \usepackage{selnolig}  % disable illegal ligatures
\fi
\IfFileExists{bookmark.sty}{\usepackage{bookmark}}{\usepackage{hyperref}}
\IfFileExists{xurl.sty}{\usepackage{xurl}}{} % add URL line breaks if available
\urlstyle{same} % disable monospaced font for URLs
\hypersetup{
  pdftitle={Economía Política Internacional {[}UCC{]}},
  pdfauthor={Sebastian Freille},
  hidelinks,
  pdfcreator={LaTeX via pandoc}}

\title{Economía Política Internacional {[}UCC{]}}
\author{Sebastian Freille}
\date{07 August, 2024}

\begin{document}
\frame{\titlepage}
\ifdefined\Shaded\renewenvironment{Shaded}{\begin{tcolorbox}[borderline west={3pt}{0pt}{shadecolor}, breakable, interior hidden, boxrule=0pt, enhanced, frame hidden, sharp corners]}{\end{tcolorbox}}\fi

\hypertarget{presentaciuxf3n-y-equipo}{%
\section{Presentación y equipo}\label{presentaciuxf3n-y-equipo}}

\begin{frame}{Bienvenida}
\protect\hypertarget{bienvenida}{}
Este es un \textbf{sitio de diapositivas de clase} de la materia
Economía Política Internacional (cátedra A) del cuarto año de las
carreras de Licenciatura en Ciencia Política y Licenciatura en
Relaciones Internacionales de la Facultad de Ciencia Política y
Relaciones Internacionales {[}CPyRRII{]} de la Universidad Católica de
Córdoba {[}UCC{]}
\end{frame}

\begin{frame}{Aspectos organizativos}
\protect\hypertarget{aspectos-organizativos}{}
En esta sección presentaremos los principales aspectos generales y
organizativos de la materia.
\end{frame}

\begin{frame}{Equipo de cátedra}
\protect\hypertarget{equipo-de-cuxe1tedra}{}
\begin{itemize}
\tightlist
\item
  \textbf{Profesor Titular: FREILLE, Sebastián}

  \begin{itemize}
  \tightlist
  \item
    Lic en Economía (UNC)
  \item
    PhD en Economía (University of Nottingham)
  \end{itemize}
\item
  Director de Proyecto de Investigación \textbf{``Grupos de interés
  especial, polarización de preferencias y políticas y recesión
  democrática''} para el período 2023-2026.
\end{itemize}
\end{frame}

\hypertarget{objetivos-y-organizaciuxf3n-del-dictado}{%
\section{Objetivos y organización del
dictado}\label{objetivos-y-organizaciuxf3n-del-dictado}}

\begin{frame}{Objetivo general}
\protect\hypertarget{objetivo-general}{}
El objetivo de la materia es estudiar y analizar la interacción entre
las relaciones económicas internacionales de los países, las
instituciones políticas y el proceso de agregación de preferencias de
múltiples actores y las políticas económicas resultantes de esa
interacción.
\end{frame}

\begin{frame}{Objetivos específicos}
\protect\hypertarget{objetivos-especuxedficos}{}
\begin{itemize}
\tightlist
\item
  Estimular y desarrollar en los estudiantes la capacidad analítica para
  el estudio de los principales fenómenos económicos internacionales.
\item
  Proponer un enfoque y abordaje positivo que permita un análisis
  riguroso de los temas
\item
  Contribuir a una mayor integración de los elementos económicos y
  políticos que interactúan en la configuración de las relaciones
  económicas internacionales
\item
  Promover en los estudiantes una visión reflexiva y crítica de los
  fenómenos estudiados
\end{itemize}
\end{frame}

\begin{frame}{Clases y dictado}
\protect\hypertarget{clases-y-dictado}{}
\begin{itemize}
\tightlist
\item
  El curso se dicta íntegramente de manera presencial \textbf{usando una
  metodología tradicional} aunque complementada con:

  \begin{enumerate}
  \tightlist
  \item
    Material digitalizado y audiovisual en aula virtual
  \item
    Actividades para realizar en aula virtual
  \end{enumerate}
\item
  Las clases serán los días \textbf{miércoles de 14:00 a 18:00} y serán
  estructuradas en 3 (tres) módulos de dictado de 1 (una) hora y 2 (dos)
  pausas de 15 minutos cada una
\end{itemize}
\end{frame}

\begin{frame}{Programa de la materia}
\protect\hypertarget{programa-de-la-materia}{}
\begin{enumerate}
\tightlist
\item
  Unidad 1: Introducción: Economía y política. Métodos y problemas
\item
  Unidad 2: Política comercial: Factores y sectores. Votantes y
  políticos. Instituciones
\item
  Unidad 3: Integración de mercados globales
\item
  Unidad 4: Economía política de la redistribución
\item
  Unidad 5: Tópicos globales de economía política
\end{enumerate}
\end{frame}

\begin{frame}{Bibliografía y textos}
\protect\hypertarget{bibliografuxeda-y-textos}{}
\begin{itemize}
\tightlist
\item
  La cátedra no sigue un texto único
\item
  Se dará una bibliografía detallada de papers y extractos de libro
\item
  Hay bibliografía en castellano pero un número importante de textos es
  en inglés
\end{itemize}
\end{frame}

\begin{frame}{Evaluación}
\protect\hypertarget{evaluaciuxf3n}{}
\begin{itemize}
\tightlist
\item
  La evaluación tendrá dos instancias: parciales y finales. Habrá 2
  (dos) parciales y se podrá recuperar hasta uno ya sea por aplazo o
  ausencia.
\item
  El alumno obtendrá la \textbf{condición de regular (R) aprobando 2
  (dos) parciales}. En caso contrario, quedará libre parciales
  (LP)/libre asistencia (LA).
\item
  La evaluación final comprenderá todos los temas del programa y la
  modalidad a definir.
\item
  Cada evaluación (parcial/final) se aprobará con el 50\% del puntaje
  total
\end{itemize}
\end{frame}

\begin{frame}{Fechas importantes}
\protect\hypertarget{fechas-importantes}{}
\begin{itemize}
\tightlist
\item
  Parcial \#1: a definir
\item
  Parcial \#2: a definir
\item
  Parcial Recuperatorio: a definir
\end{itemize}
\end{frame}

\begin{frame}{Puntos de encuentro}
\protect\hypertarget{puntos-de-encuentro}{}
\begin{enumerate}
\tightlist
\item
  Aula virtual de la materia:
  \href{https://campusvirtual.ucc.edu.ar/course/view.php?id=2037}{Aula
  virtual}
\item
  Web de S Freille:
  \href{https://sfreille.github.io/teaching/epin}{Página de Economía
  Política Internacional}
\item
  Mail y otros

  \begin{itemize}
  \tightlist
  \item
    Se responden mails y mensajes en foros (no usamos grupos de
    whatsapp)
  \end{itemize}
\end{enumerate}
\end{frame}

\begin{frame}{Resumiendo}
\protect\hypertarget{resumiendo}{}
\begin{enumerate}
\tightlist
\item
  Usen los espacios y metodologías dispuestos por la cátedra
\item
  \textbf{Dificultad media/alta} pero la filosofía de la catedra es:

  \begin{enumerate}
  \tightlist
  \item
    Explicamos los temas del programa
  \item
    Evaluamos sólo lo que explicamos
  \item
    Explicamos cómo evaluamos y cómo corregimos
  \item
    Mostramos evaluaciones y devoluciones
  \end{enumerate}
\end{enumerate}
\end{frame}



\end{document}
